\documentclass[a4paper,12pt]{article} 
\usepackage[T2A]{fontenc}			
\usepackage[utf8]{inputenc}			
\usepackage[english,russian]{babel}
\usepackage{float}
\usepackage{amsmath,amsfonts,amssymb,amsthm,mathrsfs,mathtools} 
\usepackage{cancel}
\usepackage{multirow}
\usepackage[colorlinks, linkcolor = blue]{hyperref}
\usepackage{upgreek}
\usepackage[left=2cm,right=2cm,top=2cm,bottom=3cm,bindingoffset=0cm]{geometry}
\usepackage{tikz}
\usepackage{graphicx}
\usepackage{subfig}
\usepackage{titletoc}
\usepackage{pgfplots}
\usepackage{xcolor}
\usepackage{wrapfig}
\usepackage{pgfplots}
\pgfplotsset{width=10cm,compat=1.9}

\newcommand{\n}{\hfill \break}

\begin{document}

\section*{Работа 4.3.2}	
\section*{Дифракция света на ультразвуковой волне в жидкости}
\subsection*{Киркича Андрей, Б01-202, МФТИ}
\n\n
\textbf{Цель работы: }
Изучение дифракции света на синусоидальной акустической решетке и наблюдение фазовой решетки методом темного поля.
	\n\n
	\textbf{В работе используются: }
Оптическая скамья, осветитель, два длиннофокусных объектива, кювета с жидкостью, кварцевый излучатель с микрометрическим винтом, генератор звуковой частоты, линза, вертикальная нить на рейтере, микроскоп.
\n\n


\section*{Теоретическая справка}
Свет может дифрагировать на стоячей звуковой волне в жидкости: это связано с тем, что при колебаниях создаются области повышенного и ножинного давления, в которых различен показатель преломления среды. При этом он меняется по закону
\begin{equation}
    n = n_0(1 + cos Kx),
\end{equation}
где $K = 2\pi/\Lambda$, $\Lambda$ -- длина ультразвуковой волны. При этом акустическую решетку можно считать фазовой, если выполнено соотношение 
\begin{equation}
    a \ll \left( \frac{\Lambda}{L} \right)^2,
\end{equation}
где $L$ -- толщина слоя жидкости в кювете. \\
\noindent
Также важны соотношения
\begin{equation}
    l_m = mf\frac{\lambda}{\Lambda},
\end{equation}
где $l_m$ -- расстояние между нулевым и $m$-тым максимумами дифракционной картины, $f$ -- фокусное расстояние линзы, используемой в установке, $\lambda$ -- длина используемой световой волны.
\begin{equation}
    v = \Lambda \nu
\end{equation}
При этом параметры установки $\lambda = 6400 \cdot 10^{-10}$ м, $f = 0.28$ м.

\newpage
\section*{Ход работы}
\paragraph{Определение скорости ультразвука по дифракционной картине}\n
Сначала соберем установку согласно рис. 1. Для этого используем светофильтр Ф, коллиматор К, горизонтальную щель S, линзы O$_1$ и O$_2$, микроскоп М и генератор частот Q в кювете C.

\begin{figure}[H]
    \centering
    \includegraphics[scale=0.3]{schema_1.png}
    \caption{Схема устновки для измерений по дфиракционной картине}
\end{figure}

\noindent
После юстировки установки, меняя частоту генератора, дождемся появления дифракционных полос, видимых в микроскоп. 

\noindent
Измерим положение дифракционных полос $Y_m$ в зависимости от номера полосы. Повторим измерения для нескольких частот, на которых видна дифракционная картина. Результаты занесем в табл. 1.

\begin{table}[H]
    \centering
    \caption{Координаты дифракционных максимумов при различных частотах}
    \resizebox{18cm}{!}{
    \begin{tabular}{|c|c|c|c|c|c|c|c|} \hline
        $\nu = 1.080$ МГц & & $\nu = 1.936$ МГц & & $\nu = 3.219$ МГц & & $\nu = 4.430$ МГц & \\ \hline
        $m$ & $x_m$, мкм & $m$ & $x_m$, мкм & $m$ & $x_m$, мкм & $m$ & $x_m$, мкм \\ \hline
        -3 & -144 & -2 & 152 & -1 & 264 &-1 & 16 \\ \hline 
        -2 & -28 & -1 & 412 & 0 & 640 & 0&560  \\ \hline
        -1 & 92 & 0 & 628 & 1 & 1008  & 1 &1076 \\ \hline
        0 & 228 & 1 & 852 & --- & --- & ---&---  \\ \hline
        1 & 368 & 2 & 1100 & --- &--- & ---&---  \\ \hline
        2 & 488 & ---& ---& ---&--- & ---&--- \\ \hline
        3 & 604 &--- &--- &--- &--- & ---&---  \\ \hline
    \end{tabular}
}
\end{table}

\noindent
По результам измерений построим графики, представленные на рис. 3. Здесь $\Delta x_m = l_m$ -- расстояние от нулевого максимума до $m$-того максимума.

\begin{figure}[H]
    \centering
    \includegraphics[scale=0.55]{1.png}
    \caption{Зависимость расстояния между максимумами от номера}
\end{figure}

\noindent
Из графика получаем коэффициенты наклона $l_m / m$. Затем, пользуюсь формулами (3) и (4), получаем длину волны и скорость ультразвука. Результаты в табл. 2.

\begin{table}[H]
    \centering
    \caption{Длина волны и скорость ультразвука}
    \begin{tabular}{|c|c|c|c|} \hline
        $\nu$, МГц & $\l_m / m$, мкм & $\Lambda$, мм & $v$, м/с  \\ \hline
        1.080 & 126.86 & 1.41 & 1525 \\ \hline
        1.936 & 233.61 & 0.77 & 1489 \\ \hline
        3.216 & 372.00 & 0.48 & 1550 \\ \hline
        4.430 & 530.00 & 0.34 & 1506\\ \hline
    \end{tabular}
\end{table}

\noindent
Отсюда среднее значение $v = 1518$ м/с.

\paragraph{Определение скорости ультразвука методом темного поля}
\n
Для измерений методом темного поля добавим к системе еще одну линзу, расположив ее между микроскопом и линзой O$_2$. Затем настроим микроскоп на резкое изображение сетки и премещая добавленную линзу добьемся того, чтобы блыи видны горизонтальные и вертикальные штрихи сетки. Затем закроим нулевой максимум щелью (положение нити, необходимое для этого было установлено в прошлой части работы). Меняя частоту, наблюдаем акустическую решетку. Измерим для различных частот расстояние координаты крайних хоршо видимымиых темных полом и число светлых промежутков между ними. Также рассчитаем, использую измерения, длину волны по формуле $\Lambda = 2 * (x_1 - x_0) / m$. Результаты в табл. 3.

\begin{table}[H]
    \centering
    \caption{Измерения методом темного поля}
    \begin{tabular}{|c|c|c|c|c|} \hline
        $\nu$, Мгц & $x_0$, мм & $x_1$, мм & $m$ & $\Lambda$, мм \\ \hline
        1.1459 & 5.0 & 6.4 & 2 & 1.40 \\ \hline
        1.2235 & 4.8 & 6.0 & 2 & 1.20 \\ \hline
        1.5673 & 4.4 & 5.8 & 3 & 0.93 \\ \hline
        2.0148 & 5.0 & 6.0 & 3 & 0.67 \\ \hline
        2.1216 & 4.7 & 6.0 & 4 & 0.65 \\ \hline
    \end{tabular}
\end{table}

\noindent
Построи теперь график зависимости $\Lambda$ от $1/\nu$ (рис. 4). По нему можем определить скорость ультразвука $v = 1790	 \pm 110$ м/с. 

\begin{figure}[H]
    \centering
    \includegraphics[scale=0.55]{2.png}
    \caption{Зависимости $\Lambda$ от $1/\nu$}
\end{figure}

\paragraph{Качественные наблюдения}
\n
При закрытии проволкой максимума с номером, отличным от 0, наблюдаем, что период картины не меняется, а менется лишь четкость картины. Это связано с тем, что на период на период влиет лишь расстояние между ближайшими максимума, которые формируют эту картину, а при закрытии одного любого из них, расстояние между ближайшими не меняется.

\pagebreak

\section*{Заключение}
Были проведены измерения скорости ультразвука двумя разными способами:
\begin{enumerate}
    \item По дифракционной картине. Получены значения $v = 1525$ м/с, $v = 1489$ м/с, $v = 1550$ м/с, $v = 1506$ м/с и среднее $v = 1518$ м/с.
    \item Измерения методом темного поля. Получено значение $v = (179 \pm 11) \cdot 10 $ м/с.
\end{enumerate}
Видим, что значение, полученное методом темного поля, не сходится в пределах погрешности со средним значением, как и не сходится с некоторыми отдельными значениями. Это может объясняться тем, что при измерении по дифракционной картине относительно велика погрешность измерений, так как эти измерения основываются на измерениях расстояния между дифракционными максимумами, расстояние между которыми может быть неверно измерено из-за мелкости измеряемой картины. Тем не менее, полученное значение можно считать приемлимым.

\end{document}



\begin{document}
\section*{Работа 4.4.1}	
\section*{Амплитудная дифракционная решётка (гониометр)}
\subsection*{Киркича Андрей, Б01-202, МФТИ}
\n\n
\textbf{Цель работы: }
знакомство с работой и настройкой гониометра, определение спектральных характеристик амплитудной решётки.
	\n\n
	\textbf{В работе используются: }
гониометр, дифракционная решётка, ртутная лампа.
\n\n
В работе исследуются спектр ртутной лампы и дисперсия ртутной решётки, определяются период и спектральные характеристики решётки и оценивается влияние ширины пучка на разрешающаую способность.
\n
\section*{Теоретическая справка}

Основное соотношение приближенной теории дифракционной решётки:

	\begin{equation}
	d\sin \varphi_m = m\lambda.
 	\label{main}
	\end{equation}
 \n
Угловая дисперсия $D$ характеризует угловое расстояние между близкими спектральными линиями:
	\begin{equation}
	D = \frac{d\varphi}{d\lambda} = \frac{m}{d \cos \varphi}=\frac{m}{\sqrt{d^{2}-m^{2} \lambda^{2}}}.
	\end{equation}
\n
Рассмотрим изображения спектра для двух узких спектральных линий с длинами волн $\lambda$ и $\lambda+\delta\lambda$. Для минимального значения $\lambda+\delta\lambda$, которое может быть определено по результатам измерений, вводят важнейшую характеристику спектрального прибора — разрешающую способность:

\begin{equation}
    R=\frac{\lambda}{\delta\lambda}.
\end{equation}


\section*{Экспериментальная установка}

\subsection*{Устройство гониометра}

Гониометр служит для точного измерения углов и находит широкое применение в оптических лабораториях.

\begin{center}
    \includegraphics[scale=0.2]{2023_04_02_a48ae02e429ba186bcd7g-2}
\end{center}

\begin{center}
\includegraphics[scale=0.2]{2023_04_02_a48ae02e429ba186bcd7g-2(2)}
\end{center}

\begin{center}
\includegraphics[scale=0.2]{2023_04_02_a48ae02e429ba186bcd7g-2(1)}

Оптическая схема и внешний вид гониометра

\end{center}

\subsection*{Ртутная лампа}

Характеристики спектра ртутной лампы привдены в таблице ниже.

\begin{center}
\begin{tabular}{|c|c|c|c|c|c|c|c|c|}
\hline
№ & $\mathrm{K}_{1}$ & $\mathrm{~K}_{2}$ & 1 & 2 & 3 & 4 & 5 & 6 \\
\hline
$\lambda$ нм. & 690,7 & 623,4 & 579,1 & 577,0 & 546,1 & 491,6 & 435,8 & 404,7 \\
\hline
Цвет & красн. & красн. & желт. & желт. & зелен. & голуб. & синий & фиолет. \\
\hline
Яркость & 4 & 4 & 10 & 8 & 10 & 4 & 4 & 3 \\
\hline
\end{tabular}
\end{center}

\section*{Ход работы}

Сначала были измерены углы для максимумов линий спектра ртутной лампы порядков $\pm 1$. Полученные данные приведены в Таблице \ref{angles}.

\begin{table}[H]
    \centering
    \begin{tabular}{|p{2cm}|p{4cm}|p{4cm}|}
    \hline  \centering{№ линии} & 1 порядок & -1 порядок \\ \hline
$6$   & $11^{\circ} 41^{\prime} 57^{\prime \prime}$  & $11^{\circ} 38^{\prime} 29^{\prime \prime}$  \\ \hline
$5$   & $12^{\circ} 36^{\prime} 34^{\prime \prime}$  & $12^{\circ} 33^{\prime} 37^{\prime \prime}$  \\ \hline
$4$   & $14^{\circ} 15^{\prime} 39^{\prime \prime}$  & $14^{\circ} 11^{\prime} 58^{\prime \prime}$  \\ \hline
$3$   & $15^{\circ} 52^{\prime} 33^{\prime \prime}$  & $15^{\circ} 48^{\prime} 17^{\prime \prime}$  \\ \hline
$2$   & $16^{\circ} 48^{\prime} 04^{\prime \prime}$  & $16^{\circ} 43^{\prime} 18^{\prime \prime}$  \\ \hline
$1$   & $16^{\circ} 51^{\prime} 56^{\prime \prime}$  & $16^{\circ} 47^{\prime} 01^{\prime \prime}$  \\ \hline
$K_2$ & $17^{\circ} 52^{\prime} 08^{\prime \prime}$  & $17^{\circ} 47^{\prime} 11^{\prime \prime}$  \\ \hline
$K_1$ & $18^{\circ} 12^{\prime} 43^{\prime \prime}$  & $18^{\circ} 06^{\prime } 48^{\prime \prime}$  \\ \hline
    \end{tabular}
    \caption{Углы линий спектра ртути}
    \label{angles}
\end{table}
\n
Далее для оценки угловой дисперсии решётки были измерены угловые координаты линий жёлтого дублета для всех видимых порядков спектра, положительных и отрицательных. Данные представлены в таблице \ref{yell}.

\begin{table}[H]
    \centering
    \begin{tabular}{|r|p{4cm}|p{4cm}|}
    \hline $m$ & $\varphi_{1}$ & $\varphi_{2}$ \\ \hline
$-3$  & $59^{\circ} 13^{\prime} 38^{\prime \prime}$  & $59^{\circ} 35^{\prime} 05^{\prime \prime}$  \\ \hline
$-2$  & $35^{\circ} 03^{\prime} 45^{\prime \prime}$  & $35^{\circ} 12^{\prime} 46^{\prime \prime}$  \\ \hline
$-1$  & $16^{\circ} 43^{\prime} 18^{\prime \prime}$  & $16^{\circ} 47^{\prime} 01^{\prime \prime}$  \\ \hline
$1$   & $16^{\circ} 48^{\prime} 04^{\prime \prime}$  & $16^{\circ} 51^{\prime} 56^{\prime \prime}$  \\ \hline
$2$   & $35^{\circ} 24^{\prime} 15^{\prime \prime}$  & $35^{\circ} 33^{\prime} 12^{\prime \prime}$  \\ \hline
$3$   & $60^{\circ} 41^{\prime} 58^{\prime \prime}$  & $61^{\circ} 04^{\prime}  30^{\prime \prime}$  \\ \hline
    \end{tabular}
    \caption{Углы жёлтых линий}
    \label{yell}
\end{table}
\n
Наконец, для оценки разрешающей способности спектрального прибора была измерена угловая ширина одной из линий жёлтого дублета по нулям интенсивности в первом и втором порядках. Угловая ширина составила $\Delta \varphi_1 = 1^{\prime}  30^{\prime \prime}$ и $ \Delta \varphi_2 = 1^{\prime}  53^{\prime \prime}$ соответственно.

\section*{Обработка данных}

По полученным данным был построен график на рис. \ref{angle} зависимости $\sin \varphi_m$ от длины волны. По коэффициенту наклона c использованием формулы \eqref{main} был найден период решётки $d = \frac{1}{k} = 2.04 \pm 0.05$ мкм.

\begin{figure}[H]
    \centering
    \includegraphics[scale=0.45]{sin.pdf}
    \caption{График зависимости $\sin \varphi$ от длины волны для первых максимумов}
    \label{angle}
\end{figure}
\n
Далее изучим зависимость $D$ от $m$ и сравним результат с формулой 2. Результат наглядно представлен на рисунке \ref{D}.

\begin{figure}[H]
    \centering
    \includegraphics[scale=0.45]{D.pdf}
    \caption{\centering Зависимость угловой дисперсии от порядка}
    \label{D}
\end{figure}
\n
Черная прямая - это теоретическая зависимость, построенная по параметру $d$, ранее определенному. Легко видеть, что теория очень хорошо описывает эксперимент
\n
Определим также разрешающую способность по формуле $\displaystyle R = \frac{\lambda}{\delta \lambda} \approx 680$, число рабочих штрихов решётки $N \approx 680$ и размер освещённой части $l = Nd \approx 1.36$ мм.

\section*{Заключение}

Мы исследовали спектральные линии ртути, определили шаг решётки, её угловую дисперсию, а также её разрешающую способность. Полученные результаты близки к теоретическим предсказаниям.

\end{document}

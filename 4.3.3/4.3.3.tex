\documentclass[14pt,a4paper]{article}
\usepackage[14pt]{extsizes}
\usepackage[left=1.5cm, right=1.5cm, top=1.5cm, bottom=1.5cm]{geometry}
\usepackage[utf8]{inputenc}
\usepackage[T2A]{fontenc}
\usepackage[english, russian]{babel}
\usepackage{amsmath,amsfonts,amssymb,amsthm,mathtools} 
\usepackage{amsfonts}
\usepackage{amssymb}
\usepackage{titleps}
\usepackage{hyperref}
\usepackage{float}
\usepackage{graphicx}
\usepackage{multirow}
\usepackage{hhline}
\usepackage{wrapfig}
\usepackage{tikz}
\usepackage{pgfplots}
\usepackage{xcolor}
\usepackage{subfig}
\usepackage{upgreek}
\usepackage{bm}
\usepackage{longtable}


\newcommand{\w}[1]{\text{#1}}
\newcommand{\und}[1]{\underline{#1}}
\newcommand{\img}[3]{
	\begin{figure}[H]
	\begin{center}
	\includegraphics[scale=#2]{#1}
	\end{center}
	\begin{center}
 	\textit{#3}
	\end{center}
	\end{figure}
}
\newcommand{\aw}[1]{
	\begin{center}
	\textit{#1}
	\end{center}
	\n
}
\newcommand{\be}[1]{
	\begin{center}
	\boxed{#1}
	\end{center}
}
\newcommand{\beb}[1]{
	\begin{equation}
	\boxed{#1}
	\end{equation}
}
\newcommand{\eb}[1]{
	\begin{equation}
	#1
	\end{equation}
}
\newcommand{\n}{\hfill \break}
\newcommand{\x}{\cdot}

\begin{document}
\section*{Работа 4.3.3}	
\section*{Исследование разрешающей способности микроскопа методом Аббе}
\subsection*{Киркича Андрей, Б01-202, МФТИ}
\n\n
\textbf{Цель работы: }
определение дифракционного предела разрешения объектива методом Аббе.
	\n\n
	\textbf{В работе используются: }
лазер, кассета с набором сеток разного периода, щели с микрометрическим винтом, оптичческий стол с набором рейтеров и крепёжных винтов, экран, линейка.
\n\n
В работе нужно было определить периоды сеток сначала по их спектру на удалённом экране, затем по увеличенному с помощью модели микроскопа изображению сеток на экране и, наконец, по результатам измерения разрешающей способности микроскопа; наблюдать явления саморепродукции, пространственной фильтрации и мультиплицирования.
\section*{Теоретическая справка}
Минимальное разрешаемое объективом расстояние определяется условием:
\[l_{min} \approx \frac{\lambda}{D / 2F_1}\]
\n
Координаты максимумов расчитываются по формулам:
\[d \sin \theta_x = m_x \lambda\]
\[d \sin \theta_y = m_y \lambda\]
Период решётки для измерений с помощью проекционного микроскопа:
\[d = \frac{\Delta}{n \Gamma} = \frac{\Delta}{n} \x \frac{a_1 a_2}{b_1 b_2}\]
\section*{Экспериментальная установка}
Схема экспериментальной установки представлена ниже.
\img{1.png}{3}{Рис. 1: Экспериментальная установка}
\section*{Ход работы}
Сначала мы определяли расстояние между соседними дифракционными максимумами по пространственному спектру решёток.
\begin{table}[H]
\centering
\begin{tabular}{|r||r|r||r|r||r|r|}
\hline
№ & $x$, мм & $n$ & $y$, мм & $n$ & $\Delta x$, мм & $\Delta y$, мм        \\ \hline \hline
1  & 148 $\pm$ 1 & 2 & 148 $\pm$ 1 & 2  & 74.0 $\pm$ 0.1 & 74.0 $\pm$ 0.1            \\ \hline
2  & 237 $\pm$ 1 & 8 & 206 $\pm$ 1 & 7 & 29.6 $\pm$ 0.1  & 29.4 $\pm$ 0.1  \\ \hline
3  & 266 $\pm$ 1 & 18 & 221 $\pm$ 1 & 15 & 14.8 $\pm$ 0.1  & 14.7 $\pm$ 0.1   \\ \hline
4  & 295 $\pm$ 1  & 4  & 219 $\pm$ 1  & 3  & 73.8 $\pm$ 0.1     & 73.0 $\pm$ 0.1           \\ \hline
5  & 295 $\pm$ 1  & 10 & 295 $\pm$ 1  & 10 & 29.5 $\pm$ 0.1      & 29.5 $\pm$ 0.1          \\ \hline
6  & 295 $\pm$ 1  & 20 & 295 $\pm$ 1  & 20 & 14.8 $\pm$ 0.1     & 14.8 $\pm$ 0.1         \\ \hline
\end{tabular}
\end{table}
\n
Затем мы собрали проекционный микроскоп и провели повторные измерения периодов изображений сеток.
\begin{table}[H]
\centering
\begin{tabular}{|r||r|r||r|r||r|r|}
\hline
№ & $x$, мм & $n$ & $y$, мм & $n$ & $\Delta x$, мм & $\Delta y$, мм        \\ \hline \hline
1  & -   & -  & -    & -  & -        & -               \\ \hline
2  & $50 \pm 1$  & $36 \pm 1$ & $50 \pm 1$   & $36 \pm 1$ & $1.4 \pm 0.1$ & $1.4 \pm 0.1$  \\ \hline
3  & $60 \pm 1$  & $20 \pm 1$ & $60 \pm 1$  & $20 \pm 1$ & $3.0 \pm 0.1$ & $3.0 \pm 0.1$                \\ \hline
4  & -   & -  & -    & -  & -        & -                \\ \hline
5  & -   & -  & -    & -  & -        & -                \\ \hline
6  & $62 \pm 1$  & $22 \pm 1$ & $62 \pm 1$   & $22 \pm 1$ & $2.8 \pm 0.1$ & $2.8 \pm 0.1$ \\ \hline
\end{tabular}
\end{table}
\n
У некоторых решёток расстояние между максимумами было слишком мало, чтобы измерить в миллиметровом масштабе, поэтому на соответствующих местах в таблице стоят прочерки.
\n\n
После этого мы поместили щелевую диафрагму с микрометрическим винтом в фокальную плоскость линзы и попытались определить для каждой решётки минимальный размер диафрагмы, при котором на экране ещё видно изображение сетки. Это удалось сделать только для двух решёток: 3-ей ($1.65 \pm 0.01$ мм) и 6-ой ($1.90 \pm 0.01$ мм). Остальные решётки давали сразу одномерную сетку.
\n\n
Также был проведён качественный опыт. Поворачивая щель, мы получали изображения решёток при различных ориентациях щели. Когда щель была расположена горизонтально, перекрывая первый горизонтальный максимум, мы наблюдали вертиакльную сетку, когда щель была вертикальной - наборот, горизонтальную сетку, при угле поворота в $45^o$ - диагональная сетка, ориентированная обратно.
\n\n
Затем, поменяв местами щель и сетку, мы пронаблюдали явление мультиплицирования. Уменьшение ширины щели при неизменной сетке влияло на размер изображения.
\img{3.jpg}{0.2}{Рис. 2: Явление мультиплицирования}
\section*{Обработка данных}
Мы определили синусы дифракционных углов $\theta_x, \theta_y$ по измерениям спектров (1-ый максимум) и, используя эти данные, рассчитали периоды решёток:
\begin{table}[H]
\centering
\begin{tabular}{|r||r|r||r|r|}
\hline
№ & $\sin \theta_x$     & $\sin \theta_y$ & $d_x$, мкм & $d_y$, мкм    \\ \hline \hline
1 & 0,054  & 0,054 & 9,9 & 9,9 \\ \hline
2 & 0,022  & 0,022 & 24,7 & 24,8\\ \hline
3 & 0,011 & 0,011 & 49,5 & 49,7\\ \hline
4 & 0,054 & 0,053 & 9,9 & 10\\ \hline
5 & 0,021  & 0,021 & 24,8 & 24,8\\ \hline
6 & 0,011 & 0,011 & 49,6 & 49,6\\ \hline
\end{tabular}
\end{table}
\n
Так же рассчитали периоды из увеличенных с помощью микроскопа изображений с $\Gamma = 75.5$:
\begin{table}[H]
\centering
\begin{tabular}{|r||r|r|}
\hline
№ & $d_x$, мкм & $d_y$, мкм \\ \hline \hline
2 & 18,4 & 18,4 \\ \hline
3 & 39,7 & 39,7 \\ \hline
6 & 37,3 & 37,3 \\ \hline
\end{tabular}
\end{table}
\n
Далее мы рассчитали минимальное расстояние, разрешаемое микроскопом:
\begin{table}[H]
\centering
\begin{tabular}{|r|r|r|}
\hline
№ & $D$, мм & $l_{min}$, мкм \\ \hline \hline
3 & 1,45 & 80,7 \\ \hline
6 & 1,70 & 68,8 \\ \hline
\end{tabular}
\end{table}
\n
Чтобы проверить теорию Аббе, мы построили график зависимости $d(1/D)$:
\img{2.png}{2}{Рис. 3: График зависимости $d(1/D)$}
\section*{Заключение}
В ходе данной лабораторной работы мы определили периоды дифракционных решёток двумя способами. Полученные результаты отличаются друг от друга существенно (отношение порядка $75 \%$), но имеют одинаковый порядок. Это может быть связано с приближённым характером используемой теории и неточным определением величин $a_1$ и $b_1$.
Несмотря на расхождения, нам удалось убедиться в справедливости формулы, описывающей теорию Аббе. 
\end{document}

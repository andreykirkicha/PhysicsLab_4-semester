\documentclass[a4paper,12pt]{article} 
\usepackage[T2A]{fontenc}			
\usepackage[utf8]{inputenc}			
\usepackage[english,russian]{babel}
\usepackage{float}
\usepackage{amsmath,amsfonts,amssymb,amsthm,mathrsfs,mathtools} 
\usepackage{cancel}
\usepackage{multirow}
\usepackage[colorlinks, linkcolor = blue]{hyperref}
\usepackage{upgreek}
\usepackage[left=2cm,right=2cm,top=2cm,bottom=3cm,bindingoffset=0cm]{geometry}
\usepackage{tikz}
\usepackage{graphicx}
\usepackage{subfig}
\usepackage{titletoc}
\usepackage{pgfplots}
\usepackage{xcolor}
\usepackage{wrapfig}
\usepackage{pgfplots}
\pgfplotsset{width=10cm,compat=1.9}

\newcommand{\n}{\hfill \break}

\begin{document}

\section*{Работа 4.3.2}	
\section*{Дифракция света на ультразвуковой волне в жидкости}
\subsection*{Киркича Андрей, Б01-202, МФТИ}
\n\n
\textbf{Цель работы: }
Изучение дифракции света на синусоидальной акустической решетке и наблюдение фазовой решетки методом темного поля.
	\n\n
	\textbf{В работе используются: }
Оптическая скамья, осветитель, два длиннофокусных объектива, кювета с жидкостью, кварцевый излучатель с микрометрическим винтом, генератор звуковой частоты, линза, вертикальная нить на рейтере, микроскоп.
\n\n


\section*{Теоретическая справка}
Свет может дифрагировать на стоячей звуковой волне в жидкости: это связано с тем, что при колебаниях создаются области повышенного и ножинного давления, в которых различен показатель преломления среды. При этом он меняется по закону
\begin{equation}
    n = n_0(1 + cos Kx),
\end{equation}
где $K = 2\pi/\Lambda$, $\Lambda$ -- длина ультразвуковой волны. При этом акустическую решетку можно считать фазовой, если выполнено соотношение 
\begin{equation}
    a \ll \left( \frac{\Lambda}{L} \right)^2,
\end{equation}
где $L$ -- толщина слоя жидкости в кювете. \\
\noindent
Также важны соотношения
\begin{equation}
    l_m = mf\frac{\lambda}{\Lambda},
\end{equation}
где $l_m$ -- расстояние между нулевым и $m$-тым максимумами дифракционной картины, $f$ -- фокусное расстояние линзы, используемой в установке, $\lambda$ -- длина используемой световой волны.
\begin{equation}
    v = \Lambda \nu
\end{equation}
При этом параметры установки $\lambda = 6400 \cdot 10^{-10}$ м, $f = 0.28$ м.

\newpage
\section*{Ход работы}
\paragraph{Определение скорости ультразвука по дифракционной картине}\n
Сначала соберем установку согласно рис. 1. Для этого используем светофильтр Ф, коллиматор К, горизонтальную щель S, линзы O$_1$ и O$_2$, микроскоп М и генератор частот Q в кювете C.

\begin{figure}[H]
    \centering
    \includegraphics[scale=0.3]{schema_1.png}
    \caption{Схема устновки для измерений по дфиракционной картине}
\end{figure}

\noindent
После юстировки установки, меняя частоту генератора, дождемся появления дифракционных полос, видимых в микроскоп. 

\noindent
Измерим положение дифракционных полос $Y_m$ в зависимости от номера полосы. Повторим измерения для нескольких частот, на которых видна дифракционная картина. Результаты занесем в табл. 1.

\begin{table}[H]
    \centering
    \caption{Координаты дифракционных максимумов при различных частотах}
    \resizebox{18cm}{!}{
    \begin{tabular}{|c|c|c|c|c|c|c|c|} \hline
        $\nu = 1.080$ МГц & & $\nu = 1.936$ МГц & & $\nu = 3.219$ МГц & & $\nu = 4.430$ МГц & \\ \hline
        $m$ & $x_m$, мкм & $m$ & $x_m$, мкм & $m$ & $x_m$, мкм & $m$ & $x_m$, мкм \\ \hline
        -3 & -144 & -2 & 152 & -1 & 264 &-1 & 16 \\ \hline 
        -2 & -28 & -1 & 412 & 0 & 640 & 0&560  \\ \hline
        -1 & 92 & 0 & 628 & 1 & 1008  & 1 &1076 \\ \hline
        0 & 228 & 1 & 852 & --- & --- & ---&---  \\ \hline
        1 & 368 & 2 & 1100 & --- &--- & ---&---  \\ \hline
        2 & 488 & ---& ---& ---&--- & ---&--- \\ \hline
        3 & 604 &--- &--- &--- &--- & ---&---  \\ \hline
    \end{tabular}
}
\end{table}

\noindent
По результам измерений построим графики, представленные на рис. 3. Здесь $\Delta x_m = l_m$ -- расстояние от нулевого максимума до $m$-того максимума.

\begin{figure}[H]
    \centering
    \includegraphics[scale=0.55]{1.png}
    \caption{Зависимость расстояния между максимумами от номера}
\end{figure}

\noindent
Из графика получаем коэффициенты наклона $l_m / m$. Затем, пользуюсь формулами (3) и (4), получаем длину волны и скорость ультразвука. Результаты в табл. 2.

\begin{table}[H]
    \centering
    \caption{Длина волны и скорость ультразвука}
    \begin{tabular}{|c|c|c|c|} \hline
        $\nu$, МГц & $\l_m / m$, мкм & $\Lambda$, мм & $v$, м/с  \\ \hline
        1.080 & 126.86 & 1.41 & 1525 \\ \hline
        1.936 & 233.61 & 0.77 & 1489 \\ \hline
        3.216 & 372.00 & 0.48 & 1550 \\ \hline
        4.430 & 530.00 & 0.34 & 1506\\ \hline
    \end{tabular}
\end{table}

\noindent
Отсюда среднее значение $v = 1518$ м/с.

\paragraph{Определение скорости ультразвука методом темного поля}
\n
Для измерений методом темного поля добавим к системе еще одну линзу, расположив ее между микроскопом и линзой O$_2$. Затем настроим микроскоп на резкое изображение сетки и премещая добавленную линзу добьемся того, чтобы блыи видны горизонтальные и вертикальные штрихи сетки. Затем закроим нулевой максимум щелью (положение нити, необходимое для этого было установлено в прошлой части работы). Меняя частоту, наблюдаем акустическую решетку. Измерим для различных частот расстояние координаты крайних хоршо видимымиых темных полом и число светлых промежутков между ними. Также рассчитаем, использую измерения, длину волны по формуле $\Lambda = 2 * (x_1 - x_0) / m$. Результаты в табл. 3.

\begin{table}[H]
    \centering
    \caption{Измерения методом темного поля}
    \begin{tabular}{|c|c|c|c|c|} \hline
        $\nu$, Мгц & $x_0$, мм & $x_1$, мм & $m$ & $\Lambda$, мм \\ \hline
        1.1459 & 5.0 & 6.4 & 2 & 1.40 \\ \hline
        1.2235 & 4.8 & 6.0 & 2 & 1.20 \\ \hline
        1.5673 & 4.4 & 5.8 & 3 & 0.93 \\ \hline
        2.0148 & 5.0 & 6.0 & 3 & 0.67 \\ \hline
        2.1216 & 4.7 & 6.0 & 4 & 0.65 \\ \hline
    \end{tabular}
\end{table}

\noindent
Построи теперь график зависимости $\Lambda$ от $1/\nu$ (рис. 4). По нему можем определить скорость ультразвука $v = 1790	 \pm 110$ м/с. 

\begin{figure}[H]
    \centering
    \includegraphics[scale=0.55]{2.png}
    \caption{Зависимости $\Lambda$ от $1/\nu$}
\end{figure}

\paragraph{Качественные наблюдения}
\n
При закрытии проволкой максимума с номером, отличным от 0, наблюдаем, что период картины не меняется, а менется лишь четкость картины. Это связано с тем, что на период на период влиет лишь расстояние между ближайшими максимума, которые формируют эту картину, а при закрытии одного любого из них, расстояние между ближайшими не меняется.

\pagebreak

\section*{Заключение}
Были проведены измерения скорости ультразвука двумя разными способами:
\begin{enumerate}
    \item По дифракционной картине. Получены значения $v = 1525$ м/с, $v = 1489$ м/с, $v = 1550$ м/с, $v = 1506$ м/с и среднее $v = 1518$ м/с.
    \item Измерения методом темного поля. Получено значение $v = (179 \pm 11) \cdot 10 $ м/с.
\end{enumerate}
Видим, что значение, полученное методом темного поля, не сходится в пределах погрешности со средним значением, как и не сходится с некоторыми отдельными значениями. Это может объясняться тем, что при измерении по дифракционной картине относительно велика погрешность измерений, так как эти измерения основываются на измерениях расстояния между дифракционными максимумами, расстояние между которыми может быть неверно измерено из-за мелкости измеряемой картины. Тем не менее, полученное значение можно считать приемлимым.

\end{document}



\begin{document}
\section*{Вопрос по выбору}	
\section*{Генерация второй гармоники в нелинейном кристалле}
\subsection*{Киркича Андрей, Б01-202, МФТИ}
\n\n
\textbf{Цель работы: }
изучение нелинейного оптического явления -- генерации второй гармоники.
\n\n
\textbf{В работе используются: }
лазер, нелинейный кристал LiIO$_3$, система ориентации кристалла (гониометр), система регистрации излучения.

\section*{Теоретическая справка}
При воздействии достаточно мощного светового пучка от лазера возникает смещение зарядов в атом, появляется индуцированный дипольный момент. Имеем уравнение движения:
\[m \ddot{x} = eE_0 \cos \omega t + F(x), \quad \text{где} \quad F(x) \text{ -- возвращающая сила}.\]
\n
В общем случае $F(x)$ можно разложить в ряд Тейлора. Если учитывать нелинейные члены, осциллятор становится ангармоническим. Учтём квадратичный член:
\[\ddot{x} + \omega_0^2 x = \frac{e}{m}E + \frac{F''(0)}{2m}x^2, \quad \text{где} \quad \omega_0^2 = \frac{b}{m}\]
\n
Такое уравнение можно решить сначала в нулевом приближении, а затем подставить это решение в ангармонический член:
\[x_0(t) = \frac{\frac{e}{m}E_0}{\omega_0^2 - \omega^2}\cos \omega t,\]
\[\ddot{x} + \omega_0^2 x = \frac{e}{m}E(t) + \frac{F''(0)}{2m}x_0^2(t).\]
\n
Решение будет содержать слагаемые с частотами $0$ и $2\omega$. Нелинейные колебания электронов приводят к нелинейности материального уравнения.
\n\n
Пусть волна частоты $\omega$ распространяется вдоль оси $Z$. Для диполя, расположенного в плоскости $z$, колебания с частотой $2\omega$ описываются функцией
\[X^{(2\omega)}(t, z) = A^2 \cos [2\omega (t -\frac{n(\omega)}{c} z)].\]
\n
Такой диполь излучает вторичную волну, фаза которой в точке $z' > z$ внутри нелинейной среды отличается на величину
\[2\omega \x n(2\omega) \frac{z' - z}{c} \quad \Rightarrow \quad \varphi (z') = 2\omega [t - \frac{n(2\omega)}{c} z' + (n(2\omega) - n(\omega))\frac{z}{c})].\]
\n
При выполнении условия пространственной синфазности
\[n(2\omega) - n(\omega) = 0\]
\n
все вторичные волны в точке $z'$ синфазны и амплитуда $E_0^{(2\omega)}$ пропорциональна $z'$. Это обеспечивается, если основная волна - обыкновенная, а волна второй гармоники - необыкновенная. В этом случае для отрицательного кристалла будет пересечение эллипсоида $n_e (2\omega)$ со сферой $n_o (\omega)$.
\n\n
В направлении $\Theta_0$ с оптической осью (угол синхронизма) $n_o (\omega) = n_e (2\omega)$. Угол синхронизма можно найти из системы:
\begin{equation*}
\begin{cases}
   n_o (\Theta) = \text{const}\\
   n_e (\Theta) = n_o [1 + (\frac{n_o^2}{n_e^2} - 1)\sin ^2 \Theta]^{-1/2}
 \end{cases}.
\end{equation*}
\n
В работе используется кристалл иодата лития -- отрицательный одноосный, показатели преломления для обыкновенной и необыкновенной волн представлены в таблице ниже.
\begin{table}[H]
\centering
\begin{tabular}{|r||r|r|}
\hline
$\lambda$, нм & $n_o$ & $n_e$ \\ \hline \hline
1064   & 1,8517 & 1,7168 \\ \hline
532    & 1,8978 & 1,7475 \\ \hline
\end{tabular}
\caption{Показатели преломления для обыкновенной $n_o$ и необыкновенной $n_e$ волн в кристалле иодата лития}
\end{table}
\n
Интенсивность второй гармоники пропорциональна
\eb{I^{(2\omega)} \sim \omega^4 \sin ^2 \Theta (I^{(\omega)})^2,}
\n
где $\Theta$ - угол между направлением  распространения луча и оптической осью.
\n\n
Также покажем, как влияет отклонение света от направления синхронизма на интенсивность второй гармоники.
\n\n

\img{2.jpg}{0.3}{Зависимость интенсивности второй гармоники $I^{(2\omega)}$ от угла $\Delta \Theta = \Theta - \Theta_0$ в нелинейном кристалле}
\n
Коэффициент преобразования во вторую гармонику рассчитывается по формуле:
\eb{K = \frac{\Delta I(\omega)}{I(\omega)}}

\section*{Экспериментальная установка}
Схема экспериментальной установки представлена ниже.
\img{1.jpg}{0.25}{Экспериментальная установка}

Излучение лазера 1, проходя ослабитель О и линзу-корректор Л, попадает в нелинейный кристалл НК, где частота его удваивается. Излучение удвоенной частоты попадает в фотоприёмник ФП и регистрируется осциллографом 4. Элементы 2 и 3 - питание.
\section*{Ход работы}
В начале был отъюстирован гониометр. Включив лазер и установив тефлоновый фильтр на фотоприёмник, мы наблюдали картину импульсов на экране осциллографа. Излучение лазера $\lambda = 1064$ нм имело круговую поляризацию -- это было установлено инфракрасным поляроидом.
\n\n
После градуировки ослабителя на столик гониометра мы установили нелинейный кристалл так, чтобы на тефлоновом фильтре появилось зелёное излучение. С помощью поляроидов для инфракрасного и видимого света было определено, что поляризация зелёного света -- круговая, а поляризация генерирующего излучения 1064 нм -- линейная.
\n\n
Затем мы сняли зависимость интенсивности линий второй гармоники $\lambda = 532$ нм от интенсивности возбуждающей линии $\lambda = 1064$ нм (предварительно был поставлен зелный фильтра). Полученные данные представлены в таблице ниже, зависимость отражена на графике.
\begin{table}[H]
\centering
\begin{tabular}{|r||r|r|r|r|r|}
\hline 
Ширина пучка, мм & 5,0 $\pm$ 0,1  & 3,2 $\pm$ 0,1  & 2,0 $\pm$ 0,1 & 1,2 $\pm$ 0,1 \\ \hline
$I_{1064}$, В & 45,5 $\pm$ 0,5 & 45,0 $\pm$ 0,5  & 44,5 $\pm$ 0,5 & 44,0 $\pm$ 0,5   \\ \hline
$I_{532}$, мВ & 1,60 $\pm$ 0,05  & 1,55 $\pm$ 0,05 & 1,40 $\pm$ 0,05  & 1,15 $\pm$ 0,05 	\\ \hline
\end{tabular}
\caption{Зависимость линии второй гармоники $\lambda = 532$ нм от интенсивности возбуждающей линии $\lambda = 1064 $ нм}
\end{table}

\img{plot_1.png}{0.5}{График зависимости $I_{532} = f(I_{1064})$}
\n
Затем при помощи была найдена зависимость второй гармоники от угла между направлением распространением луча и направлением синхронизма. Снятые данные и соответствующий график приведены ниже. Погрешность $\Theta$ мы взяли за $1'' \approx 0,0003$ град, у $\Delta \Theta$ -- $0,0006$ град, чтобы не загромождать таблицу, вынесем их отдельно.

\begin{table}[H]
\centering
\begin{tabular}{|r|r|r|}
\hline
$\Theta$, град         & $\Delta \Theta$, град         &  $I_{532}$, мВ    \\ \hline \hline
$\Theta_0 = $ 348,354 & 0,000        & 1,55 $\pm$ 0,05  \\ \hline
348,321 & 0,033 & 1,45 $\pm$ 0,05 \\ \hline
348,308 & 0,045 & 1,35 $\pm$ 0,05 \\ \hline
348,295 & 0,059 & 1,25 $\pm$ 0,05 \\ \hline
348,284 & 0,069 & 1,15 $\pm$ 0,05 \\ \hline
348,278 & 0,076 & 1,05 $\pm$ 0,05 \\ \hline
348,269 & 0,085 & 0,95 $\pm$ 0,05 \\ \hline
\end{tabular}
\caption{Зависимость интенсивности второй гармоники $I_{532}$ от угла $\Delta \Theta$ между направлением распространия луча $\lambda = 1064$ нм и направлением синхронизма}
\end{table}

\img{plot_2.png}{0.6}{График зависимости $I^{(2\omega)} = f(\Delta \Theta)$}
\n
В последнем пункте работы мы измерили интенсивность возбуждающей линии, прошедшей через кристалл в случаях, когда излучение $\lambda = 532$ нм и когда оно практически отсутствует (для этого поворачивали кристалл), а затем вычислили коэффициент преобразования во вторую гармонику по формуле (2):
\[K = \frac{32 - 31}{32} = 0,03 \pm 0,07.\]

\section*{Заключение}
Зависимость $I_{532} = f(I_{1064})$, полученная в первом пункте работы, визуально отличается от теоретической (1), но с учётом погрешностей через неё предположительно можно провести параболу. Расхождение может быть связано с плохим качеством системы: линза-корректор болталась на оптической скамье, отклоняясь от оси системы и изменяя интенсивность излучения. Эта особенность была замечена ближе к концу работы. Градуировка ограничителя, закреплённого на этой линзе, не была проделана заново.\n\n
График $I^{(2\omega)}(\Delta \Theta)$ хорошо повторяет теоретичкую зависимость в правой половине. Это говорит о хорошей юстировке гониометра, и можно предполагать, что левая часть зависимости также будет точно приближена экспериментальными точками.\n\n
В последнем пункте работы было получено очень приближённое значение коэффициента $K$ с погрешностью, превосходящей значение. Предложенный метод имеет крайне низкую точность, поэтому стоит использовать другие способы измерения интенсивностей.
\end{document}

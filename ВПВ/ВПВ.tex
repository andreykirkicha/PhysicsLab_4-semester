\documentclass[14pt,a4paper]{article}
\usepackage[14pt]{extsizes}
\usepackage[left=1.5cm, right=1.5cm, top=1.5cm, bottom=1.5cm]{geometry}
\usepackage[utf8]{inputenc}
\usepackage[T2A]{fontenc}
\usepackage[english, russian]{babel}
\usepackage{amsmath,amsfonts,amssymb,amsthm,mathtools} 
\usepackage{amsfonts}
\usepackage{amssymb}
\usepackage{titleps}
\usepackage{hyperref}
\usepackage{float}
\usepackage{graphicx}
\usepackage{multirow}
\usepackage{hhline}
\usepackage{wrapfig}
\usepackage{tikz}
\usepackage{pgfplots}
\usepackage{xcolor}
\usepackage{subfig}
\usepackage{upgreek}
\usepackage{bm}
\usepackage{longtable}


\newcommand{\w}[1]{\text{#1}}
\newcommand{\und}[1]{\underline{#1}}
\newcommand{\img}[3]{
	\begin{figure}[H]
	\begin{center}
	\includegraphics[scale=#2]{#1}
	\end{center}
	\begin{center}
 	\textit{#3}
	\end{center}
	\end{figure}
}
\newcommand{\aw}[1]{
	\begin{center}
	\textit{#1}
	\end{center}
	\n
}
\newcommand{\be}[1]{
	\begin{center}
	\boxed{#1}
	\end{center}
}
\newcommand{\beb}[1]{
	\begin{equation}
	\boxed{#1}
	\end{equation}
}
\newcommand{\eb}[1]{
	\begin{equation}
	#1
	\end{equation}
}
\newcommand{\n}{\hfill \break}
\newcommand{\x}{\cdot}

\begin{document}
\section*{Вопрос по выбору}	
\section*{Генерация второй гармоники в нелинейном кристалле}
\subsection*{Киркича Андрей, Б01-202, МФТИ}
\n\n
\textbf{Цель работы: }
изучение нелинейного оптического явления -- генерации второй гармоники.
\n\n
\textbf{В работе используются: }
лазер, нелинейный кристал LiIO$_3$, система ориентации кристалла (гониометр), система регистрации излучения.

\section*{Теоретическая справка}
При воздействии достаточно мощного светового пучка от лазера возникает смещение зарядов в атом, появляется индуцированный дипольный момент. Имеем уравнение движения:
\[m \ddot{x} = eE_0 \cos \omega t + F(x), \quad \text{где} \quad F(x) \text{ -- возвращающая сила}.\]
\n
В общем случае $F(x)$ можно разложить в ряд Тейлора. Если учитывать нелинейные члены, осциллятор становится ангармоническим. Учтём квадратичный член:
\[\ddot{x} + \omega_0^2 x = \frac{e}{m}E + \frac{F''(0)}{2m}x^2, \quad \text{где} \quad \omega_0^2 = \frac{b}{m}\]
\n
Такое уравнение можно решить сначала в нулевом приближении, а затем подставить это решение в ангармонический член:
\[x_0(t) = \frac{\frac{e}{m}E_0}{\omega_0^2 - \omega^2}\cos \omega t,\]
\[\ddot{x} + \omega_0^2 x = \frac{e}{m}E(t) + \frac{F''(0)}{2m}x_0^2(t).\]
\n
Решение будет содержать слагаемые с частотами $0$ и $2\omega$. Нелинейные колебания электронов приводят к нелинейности материального уравнения.
\n\n
Пусть волна частоты $\omega$ распространяется вдоль оси $Z$. Для диполя, расположенного в плоскости $z$, колебания с частотой $2\omega$ описываются функцией
\[X^{(2\omega)}(t, z) = A^2 \cos [2\omega (t -\frac{n(\omega)}{c} z)].\]
\n
Такой диполь излучает вторичную волну, фаза которой в точке $z' > z$ внутри нелинейной среды отличается на величину
\[2\omega \x n(2\omega) \frac{z' - z}{c} \quad \Rightarrow \quad \varphi (z') = 2\omega [t - \frac{n(2\omega)}{c} z' + (n(2\omega) - n(\omega))\frac{z}{c})].\]
\n
При выполнении условия пространственной синфазности
\[n(2\omega) - n(\omega) = 0\]
\n
все вторичные волны в точке $z'$ синфазны и амплитуда $E_0^{(2\omega)}$ пропорциональна $z'$. Это обеспечивается, если основная волна - обыкновенная, а волна второй гармоники - необыкновенная. В этом случае для отрицательного кристалла будет пересечение эллипсоида $n_e (2\omega)$ со сферой $n_o (\omega)$.
\n\n
В направлении $\Theta_0$ с оптической осью (угол синхронизма) $n_o (\omega) = n_e (2\omega)$. Угол синхронизма можно найти из системы:
\begin{equation*}
\begin{cases}
   n_o (\Theta) = \text{const}\\
   n_e (\Theta) = n_o [1 + (\frac{n_o^2}{n_e^2} - 1)\sin ^2 \Theta]^{-1/2}
 \end{cases}.
\end{equation*}
\n
В работе используется кристалл иодата лития -- отрицательный одноосный, показатели преломления для обыкновенной и необыкновенной волн представлены в таблице ниже.
\begin{table}[H]
\centering
\begin{tabular}{|r||r|r|}
\hline
$\lambda$, нм & $n_o$ & $n_e$ \\ \hline \hline
1064   & 1,8517 & 1,7168 \\ \hline
532    & 1,8978 & 1,7475 \\ \hline
\end{tabular}
\caption{Показатели преломления для обыкновенной $n_o$ и необыкновенной $n_e$ волн в кристалле иодата лития}
\end{table}
\n
Интенсивность второй гармоники пропорциональна
\eb{I^{(2\omega)} \sim \omega^4 \sin ^2 \Theta (I^{(\omega)})^2,}
\n
где $\Theta$ - угол между направлением  распространения луча и оптической осью.
\n\n
Также покажем, как влияет отклонение света от направления синхронизма на интенсивность второй гармоники.
\n\n

\img{2.jpg}{0.3}{Зависимость интенсивности второй гармоники $I^{(2\omega)}$ от угла $\Delta \Theta = \Theta - \Theta_0$ в нелинейном кристалле}
\n
Коэффициент преобразования во вторую гармонику рассчитывается по формуле:
\eb{K = \frac{\Delta I(\omega)}{I(\omega)}}

\section*{Экспериментальная установка}
Схема экспериментальной установки представлена ниже.
\img{1.jpg}{0.25}{Экспериментальная установка}

Излучение лазера 1, проходя ослабитель О и линзу-корректор Л, попадает в нелинейный кристалл НК, где частота его удваивается. Излучение удвоенной частоты попадает в фотоприёмник ФП и регистрируется осциллографом 4. Элементы 2 и 3 - питание.
\section*{Ход работы}
В начале был отъюстирован гониометр. Включив лазер и установив тефлоновый фильтр на фотоприёмник, мы наблюдали картину импульсов на экране осциллографа. Излучение лазера $\lambda = 1064$ нм имело круговую поляризацию -- это было установлено инфракрасным поляроидом.
\n\n
После градуировки ослабителя на столик гониометра мы установили нелинейный кристалл так, чтобы на тефлоновом фильтре появилось зелёное излучение. С помощью поляроидов для инфракрасного и видимого света было определено, что поляризация зелёного света -- круговая, а поляризация генерирующего излучения 1064 нм -- линейная.
\n\n
Затем мы сняли зависимость интенсивности линий второй гармоники $\lambda = 532$ нм от интенсивности возбуждающей линии $\lambda = 1064$ нм (предварительно был поставлен зелный фильтра). Полученные данные представлены в таблице ниже, зависимость отражена на графике.
\begin{table}[H]
\centering
\begin{tabular}{|r||r|r|r|r|r|}
\hline 
Ширина пучка, мм & 5,0 $\pm$ 0,1  & 3,2 $\pm$ 0,1  & 2,0 $\pm$ 0,1 & 1,2 $\pm$ 0,1 \\ \hline
$I_{1064}$, В & 45,5 $\pm$ 0,5 & 45,0 $\pm$ 0,5  & 44,5 $\pm$ 0,5 & 44,0 $\pm$ 0,5   \\ \hline
$I_{532}$, мВ & 1,60 $\pm$ 0,05  & 1,55 $\pm$ 0,05 & 1,40 $\pm$ 0,05  & 1,15 $\pm$ 0,05 	\\ \hline
\end{tabular}
\caption{Зависимость линии второй гармоники $\lambda = 532$ нм от интенсивности возбуждающей линии $\lambda = 1064 $ нм}
\end{table}

\img{plot_1.png}{0.5}{График зависимости $I_{532} = f(I_{1064})$}
\n
Затем при помощи была найдена зависимость второй гармоники от угла между направлением распространением луча и направлением синхронизма. Снятые данные и соответствующий график приведены ниже. Погрешность $\Theta$ мы взяли за $1'' \approx 0,0003$ град, у $\Delta \Theta$ -- $0,0006$ град, чтобы не загромождать таблицу, вынесем их отдельно.

\begin{table}[H]
\centering
\begin{tabular}{|r|r|r|}
\hline
$\Theta$, град         & $\Delta \Theta$, град         &  $I_{532}$, мВ    \\ \hline \hline
$\Theta_0 = $ 348,354 & 0,000        & 1,55 $\pm$ 0,05  \\ \hline
348,321 & 0,033 & 1,45 $\pm$ 0,05 \\ \hline
348,308 & 0,045 & 1,35 $\pm$ 0,05 \\ \hline
348,295 & 0,059 & 1,25 $\pm$ 0,05 \\ \hline
348,284 & 0,069 & 1,15 $\pm$ 0,05 \\ \hline
348,278 & 0,076 & 1,05 $\pm$ 0,05 \\ \hline
348,269 & 0,085 & 0,95 $\pm$ 0,05 \\ \hline
\end{tabular}
\caption{Зависимость интенсивности второй гармоники $I_{532}$ от угла $\Delta \Theta$ между направлением распространия луча $\lambda = 1064$ нм и направлением синхронизма}
\end{table}

\img{plot_2.png}{0.6}{График зависимости $I^{(2\omega)} = f(\Delta \Theta)$}
\n
В последнем пункте работы мы измерили интенсивность возбуждающей линии, прошедшей через кристалл в случаях, когда излучение $\lambda = 532$ нм и когда оно практически отсутствует (для этого поворачивали кристалл), а затем вычислили коэффициент преобразования во вторую гармонику по формуле (2):
\[K = \frac{32 - 31}{32} = 0,03 \pm 0,07.\]

\section*{Заключение}
Зависимость $I_{532} = f(I_{1064})$, полученная в первом пункте работы, визуально отличается от теоретической (1), но с учётом погрешностей через неё предположительно можно провести параболу. Расхождение может быть связано с плохим качеством системы: линза-корректор болталась на оптической скамье, отклоняясь от оси системы и изменяя интенсивность излучения. Эта особенность была замечена ближе к концу работы. Градуировка ограничителя, закреплённого на этой линзе, не была проделана заново.\n\n
График $I^{(2\omega)}(\Delta \Theta)$ хорошо повторяет теоретичкую зависимость в правой половине. Это говорит о хорошей юстировке гониометра, и можно предполагать, что левая часть зависимости также будет точно приближена экспериментальными точками.\n\n
В последнем пункте работы было получено очень приближённое значение коэффициента $K$ с погрешностью, превосходящей значение. Предложенный метод имеет крайне низкую точность, поэтому стоит использовать другие способы измерения интенсивностей.
\end{document}

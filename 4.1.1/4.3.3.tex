\documentclass[a4paper,12pt]{article} 
\usepackage[T2A]{fontenc}			
\usepackage[utf8]{inputenc}			
\usepackage[english,russian]{babel}
\usepackage{float}
\usepackage{amsmath,amsfonts,amssymb,amsthm,mathrsfs,mathtools} 
\usepackage{cancel}
\usepackage{multirow}
\usepackage[colorlinks, linkcolor = blue]{hyperref}
\usepackage{upgreek}
\usepackage[left=2cm,right=2cm,top=2cm,bottom=3cm,bindingoffset=0cm]{geometry}
\usepackage{tikz}
\usepackage{graphicx}
\usepackage{subfig}
\usepackage{titletoc}
\usepackage{pgfplots}
\usepackage{xcolor}
\usepackage{wrapfig}
\usepackage{pgfplots}
\pgfplotsset{width=10cm,compat=1.9}

\newcommand{\n}{\hfill \break}

\begin{document}

\section*{Работа 4.3.2}	
\section*{Дифракция света на ультразвуковой волне в жидкости}
\subsection*{Киркича Андрей, Б01-202, МФТИ}
\n\n
\textbf{Цель работы: }
Изучение дифракции света на синусоидальной акустической решетке и наблюдение фазовой решетки методом темного поля.
	\n\n
	\textbf{В работе используются: }
Оптическая скамья, осветитель, два длиннофокусных объектива, кювета с жидкостью, кварцевый излучатель с микрометрическим винтом, генератор звуковой частоты, линза, вертикальная нить на рейтере, микроскоп.
\n\n


\section*{Теоретическая справка}
Свет может дифрагировать на стоячей звуковой волне в жидкости: это связано с тем, что при колебаниях создаются области повышенного и ножинного давления, в которых различен показатель преломления среды. При этом он меняется по закону
\begin{equation}
    n = n_0(1 + cos Kx),
\end{equation}
где $K = 2\pi/\Lambda$, $\Lambda$ -- длина ультразвуковой волны. При этом акустическую решетку можно считать фазовой, если выполнено соотношение 
\begin{equation}
    a \ll \left( \frac{\Lambda}{L} \right)^2,
\end{equation}
где $L$ -- толщина слоя жидкости в кювете. \\
\noindent
Также важны соотношения
\begin{equation}
    l_m = mf\frac{\lambda}{\Lambda},
\end{equation}
где $l_m$ -- расстояние между нулевым и $m$-тым максимумами дифракционной картины, $f$ -- фокусное расстояние линзы, используемой в установке, $\lambda$ -- длина используемой световой волны.
\begin{equation}
    v = \Lambda \nu
\end{equation}
При этом параметры установки $\lambda = 6400 \cdot 10^{-10}$ м, $f = 0.28$ м.

\newpage
\section*{Ход работы}
\paragraph{Определение скорости ультразвука по дифракционной картине}\n
Сначала соберем установку согласно рис. 1. Для этого используем светофильтр Ф, коллиматор К, горизонтальную щель S, линзы O$_1$ и O$_2$, микроскоп М и генератор частот Q в кювете C.

\begin{figure}[H]
    \centering
    \includegraphics[scale=0.3]{schema_1.png}
    \caption{Схема устновки для измерений по дфиракционной картине}
\end{figure}

\noindent
После юстировки установки, меняя частоту генератора, дождемся появления дифракционных полос, видимых в микроскоп. 

\noindent
Измерим положение дифракционных полос $Y_m$ в зависимости от номера полосы. Повторим измерения для нескольких частот, на которых видна дифракционная картина. Результаты занесем в табл. 1.

\begin{table}[H]
    \centering
    \caption{Координаты дифракционных максимумов при различных частотах}
    \resizebox{18cm}{!}{
    \begin{tabular}{|c|c|c|c|c|c|c|c|} \hline
        $\nu = 1.080$ МГц & & $\nu = 1.936$ МГц & & $\nu = 3.219$ МГц & & $\nu = 4.430$ МГц & \\ \hline
        $m$ & $x_m$, мкм & $m$ & $x_m$, мкм & $m$ & $x_m$, мкм & $m$ & $x_m$, мкм \\ \hline
        -3 & -144 & -2 & 152 & -1 & 264 &-1 & 16 \\ \hline 
        -2 & -28 & -1 & 412 & 0 & 640 & 0&560  \\ \hline
        -1 & 92 & 0 & 628 & 1 & 1008  & 1 &1076 \\ \hline
        0 & 228 & 1 & 852 & --- & --- & ---&---  \\ \hline
        1 & 368 & 2 & 1100 & --- &--- & ---&---  \\ \hline
        2 & 488 & ---& ---& ---&--- & ---&--- \\ \hline
        3 & 604 &--- &--- &--- &--- & ---&---  \\ \hline
    \end{tabular}
}
\end{table}

\noindent
По результам измерений построим графики, представленные на рис. 3. Здесь $\Delta x_m = l_m$ -- расстояние от нулевого максимума до $m$-того максимума.

\begin{figure}[H]
    \centering
    \includegraphics[scale=0.55]{1.png}
    \caption{Зависимость расстояния между максимумами от номера}
\end{figure}

\noindent
Из графика получаем коэффициенты наклона $l_m / m$. Затем, пользуюсь формулами (3) и (4), получаем длину волны и скорость ультразвука. Результаты в табл. 2.

\begin{table}[H]
    \centering
    \caption{Длина волны и скорость ультразвука}
    \begin{tabular}{|c|c|c|c|} \hline
        $\nu$, МГц & $\l_m / m$, мкм & $\Lambda$, мм & $v$, м/с  \\ \hline
        1.080 & 126.86 & 1.41 & 1525 \\ \hline
        1.936 & 233.61 & 0.77 & 1489 \\ \hline
        3.216 & 372.00 & 0.48 & 1550 \\ \hline
        4.430 & 530.00 & 0.34 & 1506\\ \hline
    \end{tabular}
\end{table}

\noindent
Отсюда среднее значение $v = 1518$ м/с.

\paragraph{Определение скорости ультразвука методом темного поля}
\n
Для измерений методом темного поля добавим к системе еще одну линзу, расположив ее между микроскопом и линзой O$_2$. Затем настроим микроскоп на резкое изображение сетки и премещая добавленную линзу добьемся того, чтобы блыи видны горизонтальные и вертикальные штрихи сетки. Затем закроим нулевой максимум щелью (положение нити, необходимое для этого было установлено в прошлой части работы). Меняя частоту, наблюдаем акустическую решетку. Измерим для различных частот расстояние координаты крайних хоршо видимымиых темных полом и число светлых промежутков между ними. Также рассчитаем, использую измерения, длину волны по формуле $\Lambda = 2 * (x_1 - x_0) / m$. Результаты в табл. 3.

\begin{table}[H]
    \centering
    \caption{Измерения методом темного поля}
    \begin{tabular}{|c|c|c|c|c|} \hline
        $\nu$, Мгц & $x_0$, мм & $x_1$, мм & $m$ & $\Lambda$, мм \\ \hline
        1.1459 & 5.0 & 6.4 & 2 & 1.40 \\ \hline
        1.2235 & 4.8 & 6.0 & 2 & 1.20 \\ \hline
        1.5673 & 4.4 & 5.8 & 3 & 0.93 \\ \hline
        2.0148 & 5.0 & 6.0 & 3 & 0.67 \\ \hline
        2.1216 & 4.7 & 6.0 & 4 & 0.65 \\ \hline
    \end{tabular}
\end{table}

\noindent
Построи теперь график зависимости $\Lambda$ от $1/\nu$ (рис. 4). По нему можем определить скорость ультразвука $v = 1790	 \pm 110$ м/с. 

\begin{figure}[H]
    \centering
    \includegraphics[scale=0.55]{2.png}
    \caption{Зависимости $\Lambda$ от $1/\nu$}
\end{figure}

\paragraph{Качественные наблюдения}
\n
При закрытии проволкой максимума с номером, отличным от 0, наблюдаем, что период картины не меняется, а менется лишь четкость картины. Это связано с тем, что на период на период влиет лишь расстояние между ближайшими максимума, которые формируют эту картину, а при закрытии одного любого из них, расстояние между ближайшими не меняется.

\pagebreak

\section*{Заключение}
Были проведены измерения скорости ультразвука двумя разными способами:
\begin{enumerate}
    \item По дифракционной картине. Получены значения $v = 1525$ м/с, $v = 1489$ м/с, $v = 1550$ м/с, $v = 1506$ м/с и среднее $v = 1518$ м/с.
    \item Измерения методом темного поля. Получено значение $v = (179 \pm 11) \cdot 10 $ м/с.
\end{enumerate}
Видим, что значение, полученное методом темного поля, не сходится в пределах погрешности со средним значением, как и не сходится с некоторыми отдельными значениями. Это может объясняться тем, что при измерении по дифракционной картине относительно велика погрешность измерений, так как эти измерения основываются на измерениях расстояния между дифракционными максимумами, расстояние между которыми может быть неверно измерено из-за мелкости измеряемой картины. Тем не менее, полученное значение можно считать приемлимым.

\end{document}



\begin{document}
\section*{Работа 4.1.1/4.1.2}	
\section*{Геометрическая оптика}
\subsection*{Киркича Андрей, Б01-202, МФТИ}

\section*{Аннотация}
В данной работе мы определили фокусные расстояния нескольких линз с помощью подзорной трубы, измерили фокусные расстояния с помощью формулы тонкой линзы, методами Бесселя и Аббе, собрали и изучили подзорную трубу Кеплера, а также исследовали составную оптическую систему.
\section*{Теоретическая справка}
Для определения фокусного расстояния отрицательной линзы мы воспользовались формулой:
\[f = l - a_0,\]
где $l$ - расстояние между отрицательной и вспомогательной положительной линзами, $a_0$ - расстояние от положительной линзы до экрана.
\n\n
Формула тонкой линзы:
\[\frac{1}{f} = \frac{1}{s} + \frac{1}{L-s}.\]
\n
Формула Бесселя:
\[f = \frac{L^2-l^2}{4L}.\]
\n
Рассчёт фокусного расстояния по методу Аббе:
\[f = \frac{\Delta x'}{\frac{y_1}{y_0} - \frac{y_2}{y_0}} = \frac{\Delta x}{\frac{y_0}{y_2} - \frac{y_0}{y_1}}.\]
\n
Фокусное расстояние составной системы:
\[\frac{1}{f} = \frac{1}{f_1} + \frac{1}{f_2} - \frac{l}{f_1 f_2}.\]
\n
Оптический интервал системы мы рассчитывали как
\[\delta = l - f_{\text{об}} - f_{\text{ок}}.\]
\section*{Экспериментальная установка}
\img{1.png}{1.9}{Схемы экспериментов, 1 - определение фокусных расстояний с помощью подзорной трубы, 2 - метод Бесселя, 3 - метод Аббе, 4 - труба Кеплера, 5 - составная система}
\section*{Ход работы}
\subsection*{Определение фокусных расстояний линз с помощью подзорной трубы}
В первом эксперименте мы измеряли фокусные расстояния четырёх положительных и одной отрицательной линз с помощью подзорной трубы, добиваясь чёткого изображения перемещением линзы. Для измерений у отрицательной линзы использовалась вспомогательная короткофокусная положительная. Линзы ставились разными сторонами к источнику. Результаты измерений приведены в таблице.

\begin{table}[H]
\centering
\begin{tabular}{|r|r|r|r|r|r|}
\hline
№     & 1   & 2    & 3    & 4    & 5    \\ \hline \hline
$f_\text{пр}$, см & 7,3 $\pm$ 0,2 & 12,2 $\pm$ 0,2 & 25,5 $\pm$ 0,2  & 17,3 $\pm$ 0,2 & -9,4 $\pm$ 0,2 \\ \hline
$f_\text{обр}$, см & 7,3 $\pm$ 0,2 & 12,2 $\pm$ 0,2 & 24,5 $\pm$ 0,2 & 17,5 $\pm$ 0,2 & -9,7 $\pm$ 0,2 \\ \hline
\end{tabular}
\end{table}

\subsection*{Измерение фокусных расстояний линз по формуле тонкой линзы и
методом Бесселя}
Измерения проводились только для одной линзы (под номером 1). Мы находили два положения линзы, при которых на экране возникают чёткие действительные изображения - в одном случае увеличенное, а в другом - уменьшенное. Затем рассчитывали фокусные расстояния по приведённым выше формулам.

\begin{table}[H]
\centering
\begin{tabular}{|rrrr|rr|r|}
\hline
\multicolumn{1}{|c|}{$L$, см} &
  \multicolumn{1}{c|}{$s_1$, см} &
  \multicolumn{1}{c|}{$s_2$, см} &
  \multicolumn{1}{c|}{$l$, см} &
  \multicolumn{2}{c|}{$f_{\text{тонк}}$, см} &
  \multicolumn{1}{c|}{$f_{\text{Бесс}}$, см} \\ \hline \hline
\multicolumn{1}{|r|}{\multirow{2}{*}{62,5 $\pm$ 0,1}} &
  \multicolumn{1}{r|}{8,4 $\pm$ 0,1} &
  \multicolumn{1}{r|}{53,2 $\pm$ 0,1} &
  44,8 $\pm$ 0,2 &
  \multicolumn{1}{r|}{7,2 $\pm$ 0,4} &
  7,9 $\pm$ 0,4 &
  7,5 $\pm$ 0,5 \\ \cline{2-7} 
\multicolumn{1}{|r|}{} &
  \multicolumn{1}{r|}{8,6 $\pm$ 0,1} &
  \multicolumn{1}{r|}{54,2 $\pm$ 0,1} &
  45,6 $\pm$ 0,2 &
  \multicolumn{1}{r|}{7,4 $\pm$ 0,4} &
  7,1 $\pm$ 0,4 &
  7,3 $\pm$ 0,5 \\ \hline \hline
\multicolumn{4}{|c|}{Среднее:} & \multicolumn{1}{r|}{7,3 $\pm$ 0,4} & 7,5 $\pm$ 0,4 & 7,4 $\pm$ 0,5 \\ \hline
\end{tabular}
\end{table}

\subsection*{Измерение фокусных расстояний методом Аббе}
В этом эксперименте мы отводили осветитель от линзы, а экран - придвигали к линзе. Измеряя соответствующие смещения и размеры изображения на экране, рассчитывали фокусное расстояние.

\begin{table}[H]
\centering
\resizebox{500pt}{!}{
\begin{tabular}{|rrrrrrr|r|r|}
\hline
\multicolumn{1}{|c|}{$y_0$, мм} &
  \multicolumn{1}{c|}{$y_1$, мм} &
  \multicolumn{1}{c|}{$y_2$, мм} &
  \multicolumn{1}{c|}{$x_1$, см} &
  \multicolumn{1}{c|}{$x_2$, см} &
  \multicolumn{1}{c|}{$x_1'$, см} &
  \multicolumn{1}{c|}{$x_2'$, см} &
  \multicolumn{1}{c|}{$f'$, см} &
  \multicolumn{1}{c|}{$f$, см} \\ \hline
\multicolumn{1}{|r|}{1 $\pm$ 1} &
  \multicolumn{1}{r|}{3$\pm$ 1} &
  \multicolumn{1}{r|}{1,2 $\pm$ 1,0} &
  \multicolumn{1}{r|}{9,5 $\pm$ 0,1} &
  \multicolumn{1}{r|}{12,7 $\pm$ 0,1} &
  \multicolumn{1}{r|}{33,3 $\pm$ 0,1} &
  18,2 $\pm$ 0,1 &
  8,3 $\pm$ 0,7 &
  6,4 $\pm$ 0,7 \\ \hline
\multicolumn{1}{|r|}{1 $\pm$ 1} &
  \multicolumn{1}{r|}{4 $\pm$ 1} &
  \multicolumn{1}{r|}{0,8 $\pm$ 1,0} &
  \multicolumn{1}{r|}{9,3 $\pm$ 0,1} &
  \multicolumn{1}{r|}{15,5 $\pm$ 0,1} &
  \multicolumn{1}{r|}{36,7 $\pm$ 0,1} &
  14,2 $\pm$ 0,1 &
  7,0 $\pm$ 0,7 &
  6,2 $\pm$ 0,7 \\ \hline
\multicolumn{1}{|r|}{1 $\pm$ 1} &
  \multicolumn{1}{r|}{3 $\pm$ 1} &
  \multicolumn{1}{r|}{1,2 $\pm$ 1,0} &
  \multicolumn{1}{r|}{9,7 $\pm$ 0,1} &
  \multicolumn{1}{r|}{12,5 $\pm$ 0,1} &
  \multicolumn{1}{r|}{35,3 $\pm$ 0,1} &
  18,8 $\pm$ 0,1 &
  9,1 $\pm$ 0,7 &
  5,6 $\pm$ 0,7 \\ \hline \hline
\multicolumn{7}{|c|}{С р е д н е е :} &
  8,1 $\pm$ 0,7 &
  6,1 $\pm$ 0,7 \\ \hline
\end{tabular}
}
\end{table}

\n
Общая таблица наглядно отражает результаты измерений фокусного расстояния всеми методами:
\begin{table}[H]
\centering
\begin{tabular}{|r|r|r|r|r|r|}
\hline
  & Ф.Т.Л. ($s_1$) & Ф.Т.Л. ($s_2$) & Бессель  & Аббе ($\Delta x$) & Аббе ($\Delta x'$) \\ \hline
$f$, см & 7,3  $\pm$ 0,4  & 7,5  $\pm$ 0,4 & 7,4  $\pm$ 0,5 & 6,1 $\pm$ 0,7 & 8,1 $\pm$ 0,7 \\ \hline
\end{tabular}
\end{table}

\subsection*{Сборка и изучение подзорной трубы Кеплера}
С помощью подзорной трубы мы определили, что в поле зрения на размере окулярной риски укладывается 1.5 ячейки сетки изображения. Затем мы собрали схему из коллиматора (линза 4), объектива (линза 3) и окуляра (линза 1). Снова измерив угловой размер изображения ячейки, мы нашли угловое увеличение трубы: $\gamma_{\text{эксп}} = \frac{\alpha}{\alpha_0} = 3 \pm 1$. Мы также рассчитали увеличение отношением диаметров светового пятна перед объективом и за окуляром: $\gamma = \frac{D_{\text{об}}}{D_{\text{ок}}} = \frac{4,5 \text{ см}}{3 \text{ см}} = 1.5 \pm 0.9$.

\subsection*{Изучение составной оптической системы}
В качестве составной оптической схемы мы использовали две линзы (1 и 5), расположенные на фиксированном расстоянии друг от друга ($4.2$ см). Фокусное расстояние системы оценивали по формуле, приведённой в теоретической справке: $f \approx 10.8$ см. Затем провели расчёт фокусного расстояния и оптического интервала по методу Бесселя.

\begin{table}[H]
\centering
\begin{tabular}{|r|r|r|r|r|}
\hline
$x_1$, см & $x_2$, см & $L$, см & $l$, см & $f$, см        \\ \hline \hline
18,5  $\pm$ 0,1  & 77,5  $\pm$ 0,1  & 90,0 $\pm$ 0,1    & 59,0 $\pm$ 0,1    & 12,8  $\pm$ 0,5 \\ \hline
19,7 $\pm$ 0,1   & 68,3 $\pm$ 0,1   & 81,0 $\pm$ 0,1    & 48,6 $\pm$ 0,1  & 12,9 $\pm$ 0,5    \\ \hline
21,2 $\pm$ 0,1   & 53,2 $\pm$ 0,1   & 67,5 $\pm$ 0,1  & 32,0 $\pm$ 0,1    & 13,0 $\pm$ 0,5 \\ \hline
11,8 $\pm$ 0,1   & 44,5 $\pm$ 0,1   & 60,0 $\pm$ 0,1    & 32,7 $\pm$ 0,1  & 10,5 $\pm$ 0,5 \\ \hline
23,5 $\pm$ 0,1   & 38,7 $\pm$ 0,1   & 55,5 $\pm$ 0,1  & 15,2 $\pm$ 0,1  & 12,8 $\pm$ 0,5 \\ \hline
\end{tabular}
\end{table}

\n
Оптический интервал равен $\delta = 6.4 \pm 0.3$ см.

\n
В переменных $(\frac{l^2}{L - \delta}, \quad L - \delta - 4f)$ расчётная формула для метода Бесселя $l^2 = (L - \delta) (L - \delta - 4f)$ даёт линейную зависимость:

\img{plot_1.png}{0.7}{График, отражающий зависимость в формуле метода Бесселя}

\section*{Выводы}
Фокусные расстояния, определённые разными способами, сильно разнятся. Метод Аббе даёт наименьшую точность. Это связано с тем, что в методе Аббе мы оценивали чёткость маленькой картинки на экране на глаз, а смещение могло сильно варьироваться. Метод Бесселя даёт значения, хорошо приближенные к теоретическим рассчётам по формуле тонкой линзы. Формулу тонкой линзы в этом опыте действительно можно применять, потому что фокусные расстояния с обеих сторон выбранной нами линзы одинаковы.\n\n
Собранная нами труба Кеплера давала увеличение порядка $\gamma \approx 3$, исходя из оценки угловых размеров ячеек. Это немного расходится с теоретическим расчётом ($\gamma_{\text{теор}} \approx 3.42$). Это можно связать с двумя причинами: во-первых, неточное определение фокусных расстояний составляющих линз (которое производилось на глаз) и, как следствие, отклонение в теоретическом значении; во-вторых, неточная оценка угловых размеров объекта (которая тоже производилась на глаз). Существенный вклад в качество эксперимента привнесла бы аппаратная оценка этих параметров. Также мы рассчитали увеличение на основе диаметров светового пятна перед объективом и за окуляром: $\gamma = 1.5$. Такое сильное расхождение с полученными данными связано с очень грубой оценкой диаметров: край световых пятен был размыт.\n\n
В последней части эксперимента мы изучали составную оптическую систему. Теоретическая оценка фокусного расстояния такой системы расходится с экспериментальнами данными. Вероятно, это связано с относительно большим расстоянием между двумя линзами, составляющими систему.

\end{document}

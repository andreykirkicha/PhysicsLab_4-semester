\documentclass[14pt,a4paper]{article}
\usepackage[14pt]{extsizes}
\usepackage[left=1.5cm, right=1.5cm, top=1.5cm, bottom=1.5cm]{geometry}
\usepackage[utf8]{inputenc}
\usepackage[T2A]{fontenc}
\usepackage[english, russian]{babel}
\usepackage{amsmath,amsfonts,amssymb,amsthm,mathtools} 
\usepackage{amsfonts}
\usepackage{amssymb}
\usepackage{titleps}
\usepackage{hyperref}
\usepackage{float}
\usepackage{graphicx}
\usepackage{multirow}
\usepackage{hhline}
\usepackage{wrapfig}
\usepackage{tikz}
\usepackage{pgfplots}
\usepackage{xcolor}
\usepackage{subfig}
\usepackage{upgreek}
\usepackage{bm}
\usepackage{longtable}


\newcommand{\w}[1]{\text{#1}}
\newcommand{\und}[1]{\underline{#1}}
\newcommand{\img}[3]{
	\begin{figure}[H]
	\begin{center}
	\includegraphics[scale=#2]{#1}
	\end{center}
	\begin{center}
 	\textit{#3}
	\end{center}
	\end{figure}
}
\newcommand{\aw}[1]{
	\begin{center}
	\textit{#1}
	\end{center}
	\n
}
\newcommand{\be}[1]{
	\begin{center}
	\boxed{#1}
	\end{center}
}
\newcommand{\beb}[1]{
	\begin{equation}
	\boxed{#1}
	\end{equation}
}
\newcommand{\eb}[1]{
	\begin{equation}
	#1
	\end{equation}
}
\newcommand{\n}{\hfill \break}
\newcommand{\x}{\cdot}

\begin{document}
\section*{Работа 4.1.1/4.1.2}	
\section*{Геометрическая оптика}
\subsection*{Киркича Андрей, Б01-202, МФТИ}

\section*{Аннотация}
В данной работе мы определили фокусные расстояния нескольких линз с помощью подзорной трубы, измерили фокусные расстояния с помощью формулы тонкой линзы, методами Бесселя и Аббе, собрали и изучили подзорную трубу Кеплера, а также исследовали составную оптическую систему.
\section*{Теоретическая справка}
Для определения фокусного расстояния отрицательной линзы мы воспользовались формулой:
\[f = l - a_0,\]
где $l$ - расстояние между отрицательной и вспомогательной положительной линзами, $a_0$ - расстояние от положительной линзы до экрана.
\n\n
Формула тонкой линзы:
\[\frac{1}{f} = \frac{1}{s} + \frac{1}{L-s}.\]
\n
Формула Бесселя:
\[f = \frac{L^2-l^2}{4L}.\]
\n
Рассчёт фокусного расстояния по методу Аббе:
\[f = \frac{\Delta x'}{\frac{y_1}{y_0} - \frac{y_2}{y_0}} = \frac{\Delta x}{\frac{y_0}{y_2} - \frac{y_0}{y_1}}.\]
\n
Фокусное расстояние составной системы:
\[\frac{1}{f} = \frac{1}{f_1} + \frac{1}{f_2} - \frac{l}{f_1 f_2}.\]
\n
Оптический интервал системы мы рассчитывали как
\[\delta = l - f_{\text{об}} - f_{\text{ок}}.\]
\section*{Экспериментальная установка}
\img{1.png}{1.9}{Схемы экспериментов, 1 - определение фокусных расстояний с помощью подзорной трубы, 2 - метод Бесселя, 3 - метод Аббе, 4 - труба Кеплера, 5 - составная система}
\section*{Ход работы}
\subsection*{Определение фокусных расстояний линз с помощью подзорной трубы}
В первом эксперименте мы измеряли фокусные расстояния четырёх положительных и одной отрицательной линз с помощью подзорной трубы, добиваясь чёткого изображения перемещением линзы. Для измерений у отрицательной линзы использовалась вспомогательная короткофокусная положительная. Линзы ставились разными сторонами к источнику. Результаты измерений приведены в таблице.

\begin{table}[H]
\centering
\begin{tabular}{|r|r|r|r|r|r|}
\hline
№     & 1   & 2    & 3    & 4    & 5    \\ \hline \hline
$f_\text{пр}$, см & 7,3 $\pm$ 0,2 & 12,2 $\pm$ 0,2 & 25,5 $\pm$ 0,2  & 17,3 $\pm$ 0,2 & -9,4 $\pm$ 0,2 \\ \hline
$f_\text{обр}$, см & 7,3 $\pm$ 0,2 & 12,2 $\pm$ 0,2 & 24,5 $\pm$ 0,2 & 17,5 $\pm$ 0,2 & -9,7 $\pm$ 0,2 \\ \hline
\end{tabular}
\end{table}

\subsection*{Измерение фокусных расстояний линз по формуле тонкой линзы и
методом Бесселя}
Измерения проводились только для одной линзы (под номером 1). Мы находили два положения линзы, при которых на экране возникают чёткие действительные изображения - в одном случае увеличенное, а в другом - уменьшенное. Затем рассчитывали фокусные расстояния по приведённым выше формулам.

\begin{table}[H]
\centering
\begin{tabular}{|rrrr|rr|r|}
\hline
\multicolumn{1}{|c|}{$L$, см} &
  \multicolumn{1}{c|}{$s_1$, см} &
  \multicolumn{1}{c|}{$s_2$, см} &
  \multicolumn{1}{c|}{$l$, см} &
  \multicolumn{2}{c|}{$f_{\text{тонк}}$, см} &
  \multicolumn{1}{c|}{$f_{\text{Бесс}}$, см} \\ \hline \hline
\multicolumn{1}{|r|}{\multirow{2}{*}{62,5 $\pm$ 0,1}} &
  \multicolumn{1}{r|}{8,4 $\pm$ 0,1} &
  \multicolumn{1}{r|}{53,2 $\pm$ 0,1} &
  44,8 $\pm$ 0,2 &
  \multicolumn{1}{r|}{7,2 $\pm$ 0,4} &
  7,9 $\pm$ 0,4 &
  7,5 $\pm$ 0,5 \\ \cline{2-7} 
\multicolumn{1}{|r|}{} &
  \multicolumn{1}{r|}{8,6 $\pm$ 0,1} &
  \multicolumn{1}{r|}{54,2 $\pm$ 0,1} &
  45,6 $\pm$ 0,2 &
  \multicolumn{1}{r|}{7,4 $\pm$ 0,4} &
  7,1 $\pm$ 0,4 &
  7,3 $\pm$ 0,5 \\ \hline \hline
\multicolumn{4}{|c|}{Среднее:} & \multicolumn{1}{r|}{7,3 $\pm$ 0,4} & 7,5 $\pm$ 0,4 & 7,4 $\pm$ 0,5 \\ \hline
\end{tabular}
\end{table}

\subsection*{Измерение фокусных расстояний методом Аббе}
В этом эксперименте мы отводили осветитель от линзы, а экран - придвигали к линзе. Измеряя соответствующие смещения и размеры изображения на экране, рассчитывали фокусное расстояние.

\begin{table}[H]
\centering
\resizebox{500pt}{!}{
\begin{tabular}{|rrrrrrr|r|r|}
\hline
\multicolumn{1}{|c|}{$y_0$, мм} &
  \multicolumn{1}{c|}{$y_1$, мм} &
  \multicolumn{1}{c|}{$y_2$, мм} &
  \multicolumn{1}{c|}{$x_1$, см} &
  \multicolumn{1}{c|}{$x_2$, см} &
  \multicolumn{1}{c|}{$x_1'$, см} &
  \multicolumn{1}{c|}{$x_2'$, см} &
  \multicolumn{1}{c|}{$f'$, см} &
  \multicolumn{1}{c|}{$f$, см} \\ \hline
\multicolumn{1}{|r|}{1 $\pm$ 1} &
  \multicolumn{1}{r|}{3$\pm$ 1} &
  \multicolumn{1}{r|}{1,2 $\pm$ 1,0} &
  \multicolumn{1}{r|}{9,5 $\pm$ 0,1} &
  \multicolumn{1}{r|}{12,7 $\pm$ 0,1} &
  \multicolumn{1}{r|}{33,3 $\pm$ 0,1} &
  18,2 $\pm$ 0,1 &
  8,3 $\pm$ 0,7 &
  6,4 $\pm$ 0,7 \\ \hline
\multicolumn{1}{|r|}{1 $\pm$ 1} &
  \multicolumn{1}{r|}{4 $\pm$ 1} &
  \multicolumn{1}{r|}{0,8 $\pm$ 1,0} &
  \multicolumn{1}{r|}{9,3 $\pm$ 0,1} &
  \multicolumn{1}{r|}{15,5 $\pm$ 0,1} &
  \multicolumn{1}{r|}{36,7 $\pm$ 0,1} &
  14,2 $\pm$ 0,1 &
  7,0 $\pm$ 0,7 &
  6,2 $\pm$ 0,7 \\ \hline
\multicolumn{1}{|r|}{1 $\pm$ 1} &
  \multicolumn{1}{r|}{3 $\pm$ 1} &
  \multicolumn{1}{r|}{1,2 $\pm$ 1,0} &
  \multicolumn{1}{r|}{9,7 $\pm$ 0,1} &
  \multicolumn{1}{r|}{12,5 $\pm$ 0,1} &
  \multicolumn{1}{r|}{35,3 $\pm$ 0,1} &
  18,8 $\pm$ 0,1 &
  9,1 $\pm$ 0,7 &
  5,6 $\pm$ 0,7 \\ \hline \hline
\multicolumn{7}{|c|}{С р е д н е е :} &
  8,1 $\pm$ 0,7 &
  6,1 $\pm$ 0,7 \\ \hline
\end{tabular}
}
\end{table}

\n
Общая таблица наглядно отражает результаты измерений фокусного расстояния всеми методами:
\begin{table}[H]
\centering
\begin{tabular}{|r|r|r|r|r|r|}
\hline
  & Ф.Т.Л. ($s_1$) & Ф.Т.Л. ($s_2$) & Бессель  & Аббе ($\Delta x$) & Аббе ($\Delta x'$) \\ \hline
$f$, см & 7,3  $\pm$ 0,4  & 7,5  $\pm$ 0,4 & 7,4  $\pm$ 0,5 & 6,1 $\pm$ 0,7 & 8,1 $\pm$ 0,7 \\ \hline
\end{tabular}
\end{table}

\subsection*{Сборка и изучение подзорной трубы Кеплера}
С помощью подзорной трубы мы определили, что в поле зрения на размере окулярной риски укладывается 1.5 ячейки сетки изображения. Затем мы собрали схему из коллиматора (линза 4), объектива (линза 3) и окуляра (линза 1). Снова измерив угловой размер изображения ячейки, мы нашли угловое увеличение трубы: $\gamma_{\text{эксп}} = \frac{\alpha}{\alpha_0} = 3 \pm 1$. Мы также рассчитали увеличение отношением диаметров светового пятна перед объективом и за окуляром: $\gamma = \frac{D_{\text{об}}}{D_{\text{ок}}} = \frac{4,5 \text{ см}}{3 \text{ см}} = 1.5 \pm 0.9$.

\subsection*{Изучение составной оптической системы}
В качестве составной оптической схемы мы использовали две линзы (1 и 5), расположенные на фиксированном расстоянии друг от друга ($4.2$ см). Фокусное расстояние системы оценивали по формуле, приведённой в теоретической справке: $f \approx 10.8$ см. Затем провели расчёт фокусного расстояния и оптического интервала по методу Бесселя.

\begin{table}[H]
\centering
\begin{tabular}{|r|r|r|r|r|}
\hline
$x_1$, см & $x_2$, см & $L$, см & $l$, см & $f$, см        \\ \hline \hline
18,5  $\pm$ 0,1  & 77,5  $\pm$ 0,1  & 90,0 $\pm$ 0,1    & 59,0 $\pm$ 0,1    & 12,8  $\pm$ 0,5 \\ \hline
19,7 $\pm$ 0,1   & 68,3 $\pm$ 0,1   & 81,0 $\pm$ 0,1    & 48,6 $\pm$ 0,1  & 12,9 $\pm$ 0,5    \\ \hline
21,2 $\pm$ 0,1   & 53,2 $\pm$ 0,1   & 67,5 $\pm$ 0,1  & 32,0 $\pm$ 0,1    & 13,0 $\pm$ 0,5 \\ \hline
11,8 $\pm$ 0,1   & 44,5 $\pm$ 0,1   & 60,0 $\pm$ 0,1    & 32,7 $\pm$ 0,1  & 10,5 $\pm$ 0,5 \\ \hline
23,5 $\pm$ 0,1   & 38,7 $\pm$ 0,1   & 55,5 $\pm$ 0,1  & 15,2 $\pm$ 0,1  & 12,8 $\pm$ 0,5 \\ \hline
\end{tabular}
\end{table}

\n
Оптический интервал равен $\delta = 6.4 \pm 0.3$ см.

\n
В переменных $(\frac{l^2}{L - \delta}, \quad L - \delta - 4f)$ расчётная формула для метода Бесселя $l^2 = (L - \delta) (L - \delta - 4f)$ даёт линейную зависимость:

\img{plot_1.png}{0.7}{График, отражающий зависимость в формуле метода Бесселя}

\section*{Выводы}
Фокусные расстояния, определённые разными способами, сильно разнятся. Метод Аббе даёт наименьшую точность. Это связано с тем, что в методе Аббе мы оценивали чёткость маленькой картинки на экране на глаз, а смещение могло сильно варьироваться. Метод Бесселя даёт значения, хорошо приближенные к теоретическим рассчётам по формуле тонкой линзы. Формулу тонкой линзы в этом опыте действительно можно применять, потому что фокусные расстояния с обеих сторон выбранной нами линзы одинаковы.\n\n
Собранная нами труба Кеплера давала увеличение порядка $\gamma \approx 3$, исходя из оценки угловых размеров ячеек. Это немного расходится с теоретическим расчётом ($\gamma_{\text{теор}} \approx 3.42$). Это можно связать с двумя причинами: во-первых, неточное определение фокусных расстояний составляющих линз (которое производилось на глаз) и, как следствие, отклонение в теоретическом значении; во-вторых, неточная оценка угловых размеров объекта (которая тоже производилась на глаз). Существенный вклад в качество эксперимента привнесла бы аппаратная оценка этих параметров. Также мы рассчитали увеличение на основе диаметров светового пятна перед объективом и за окуляром: $\gamma = 1.5$. Такое сильное расхождение с полученными данными связано с очень грубой оценкой диаметров: край световых пятен был размыт.\n\n
В последней части эксперимента мы изучали составную оптическую систему. Теоретическая оценка фокусного расстояния такой системы расходится с экспериментальнами данными. Вероятно, это связано с относительно большим расстоянием между двумя линзами, составляющими систему.

\end{document}

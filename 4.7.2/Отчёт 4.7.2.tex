\documentclass[a4paper,12pt]{article} 
\usepackage[T2A]{fontenc}			
\usepackage[utf8]{inputenc}			
\usepackage[english,russian]{babel}
\usepackage{float}
\usepackage{amsmath,amsfonts,amssymb,amsthm,mathrsfs,mathtools} 
\usepackage{cancel}
\usepackage{multirow}
\usepackage[colorlinks, linkcolor = blue]{hyperref}
\usepackage{upgreek}
\usepackage[left=2cm,right=2cm,top=2cm,bottom=3cm,bindingoffset=0cm]{geometry}
\usepackage{tikz}
\usepackage{graphicx}
\usepackage{subfig}
\usepackage{titletoc}
\usepackage{pgfplots}
\usepackage{xcolor}
\usepackage{wrapfig}
\usepackage{pgfplots}
\pgfplotsset{width=10cm,compat=1.9}

\newcommand{\n}{\hfill \break}

\begin{document}

\section*{Работа 4.3.2}	
\section*{Дифракция света на ультразвуковой волне в жидкости}
\subsection*{Киркича Андрей, Б01-202, МФТИ}
\n\n
\textbf{Цель работы: }
Изучение дифракции света на синусоидальной акустической решетке и наблюдение фазовой решетки методом темного поля.
	\n\n
	\textbf{В работе используются: }
Оптическая скамья, осветитель, два длиннофокусных объектива, кювета с жидкостью, кварцевый излучатель с микрометрическим винтом, генератор звуковой частоты, линза, вертикальная нить на рейтере, микроскоп.
\n\n


\section*{Теоретическая справка}
Свет может дифрагировать на стоячей звуковой волне в жидкости: это связано с тем, что при колебаниях создаются области повышенного и ножинного давления, в которых различен показатель преломления среды. При этом он меняется по закону
\begin{equation}
    n = n_0(1 + cos Kx),
\end{equation}
где $K = 2\pi/\Lambda$, $\Lambda$ -- длина ультразвуковой волны. При этом акустическую решетку можно считать фазовой, если выполнено соотношение 
\begin{equation}
    a \ll \left( \frac{\Lambda}{L} \right)^2,
\end{equation}
где $L$ -- толщина слоя жидкости в кювете. \\
\noindent
Также важны соотношения
\begin{equation}
    l_m = mf\frac{\lambda}{\Lambda},
\end{equation}
где $l_m$ -- расстояние между нулевым и $m$-тым максимумами дифракционной картины, $f$ -- фокусное расстояние линзы, используемой в установке, $\lambda$ -- длина используемой световой волны.
\begin{equation}
    v = \Lambda \nu
\end{equation}
При этом параметры установки $\lambda = 6400 \cdot 10^{-10}$ м, $f = 0.28$ м.

\newpage
\section*{Ход работы}
\paragraph{Определение скорости ультразвука по дифракционной картине}\n
Сначала соберем установку согласно рис. 1. Для этого используем светофильтр Ф, коллиматор К, горизонтальную щель S, линзы O$_1$ и O$_2$, микроскоп М и генератор частот Q в кювете C.

\begin{figure}[H]
    \centering
    \includegraphics[scale=0.3]{schema_1.png}
    \caption{Схема устновки для измерений по дфиракционной картине}
\end{figure}

\noindent
После юстировки установки, меняя частоту генератора, дождемся появления дифракционных полос, видимых в микроскоп. 

\noindent
Измерим положение дифракционных полос $Y_m$ в зависимости от номера полосы. Повторим измерения для нескольких частот, на которых видна дифракционная картина. Результаты занесем в табл. 1.

\begin{table}[H]
    \centering
    \caption{Координаты дифракционных максимумов при различных частотах}
    \resizebox{18cm}{!}{
    \begin{tabular}{|c|c|c|c|c|c|c|c|} \hline
        $\nu = 1.080$ МГц & & $\nu = 1.936$ МГц & & $\nu = 3.219$ МГц & & $\nu = 4.430$ МГц & \\ \hline
        $m$ & $x_m$, мкм & $m$ & $x_m$, мкм & $m$ & $x_m$, мкм & $m$ & $x_m$, мкм \\ \hline
        -3 & -144 & -2 & 152 & -1 & 264 &-1 & 16 \\ \hline 
        -2 & -28 & -1 & 412 & 0 & 640 & 0&560  \\ \hline
        -1 & 92 & 0 & 628 & 1 & 1008  & 1 &1076 \\ \hline
        0 & 228 & 1 & 852 & --- & --- & ---&---  \\ \hline
        1 & 368 & 2 & 1100 & --- &--- & ---&---  \\ \hline
        2 & 488 & ---& ---& ---&--- & ---&--- \\ \hline
        3 & 604 &--- &--- &--- &--- & ---&---  \\ \hline
    \end{tabular}
}
\end{table}

\noindent
По результам измерений построим графики, представленные на рис. 3. Здесь $\Delta x_m = l_m$ -- расстояние от нулевого максимума до $m$-того максимума.

\begin{figure}[H]
    \centering
    \includegraphics[scale=0.55]{1.png}
    \caption{Зависимость расстояния между максимумами от номера}
\end{figure}

\noindent
Из графика получаем коэффициенты наклона $l_m / m$. Затем, пользуюсь формулами (3) и (4), получаем длину волны и скорость ультразвука. Результаты в табл. 2.

\begin{table}[H]
    \centering
    \caption{Длина волны и скорость ультразвука}
    \begin{tabular}{|c|c|c|c|} \hline
        $\nu$, МГц & $\l_m / m$, мкм & $\Lambda$, мм & $v$, м/с  \\ \hline
        1.080 & 126.86 & 1.41 & 1525 \\ \hline
        1.936 & 233.61 & 0.77 & 1489 \\ \hline
        3.216 & 372.00 & 0.48 & 1550 \\ \hline
        4.430 & 530.00 & 0.34 & 1506\\ \hline
    \end{tabular}
\end{table}

\noindent
Отсюда среднее значение $v = 1518$ м/с.

\paragraph{Определение скорости ультразвука методом темного поля}
\n
Для измерений методом темного поля добавим к системе еще одну линзу, расположив ее между микроскопом и линзой O$_2$. Затем настроим микроскоп на резкое изображение сетки и премещая добавленную линзу добьемся того, чтобы блыи видны горизонтальные и вертикальные штрихи сетки. Затем закроим нулевой максимум щелью (положение нити, необходимое для этого было установлено в прошлой части работы). Меняя частоту, наблюдаем акустическую решетку. Измерим для различных частот расстояние координаты крайних хоршо видимымиых темных полом и число светлых промежутков между ними. Также рассчитаем, использую измерения, длину волны по формуле $\Lambda = 2 * (x_1 - x_0) / m$. Результаты в табл. 3.

\begin{table}[H]
    \centering
    \caption{Измерения методом темного поля}
    \begin{tabular}{|c|c|c|c|c|} \hline
        $\nu$, Мгц & $x_0$, мм & $x_1$, мм & $m$ & $\Lambda$, мм \\ \hline
        1.1459 & 5.0 & 6.4 & 2 & 1.40 \\ \hline
        1.2235 & 4.8 & 6.0 & 2 & 1.20 \\ \hline
        1.5673 & 4.4 & 5.8 & 3 & 0.93 \\ \hline
        2.0148 & 5.0 & 6.0 & 3 & 0.67 \\ \hline
        2.1216 & 4.7 & 6.0 & 4 & 0.65 \\ \hline
    \end{tabular}
\end{table}

\noindent
Построи теперь график зависимости $\Lambda$ от $1/\nu$ (рис. 4). По нему можем определить скорость ультразвука $v = 1790	 \pm 110$ м/с. 

\begin{figure}[H]
    \centering
    \includegraphics[scale=0.55]{2.png}
    \caption{Зависимости $\Lambda$ от $1/\nu$}
\end{figure}

\paragraph{Качественные наблюдения}
\n
При закрытии проволкой максимума с номером, отличным от 0, наблюдаем, что период картины не меняется, а менется лишь четкость картины. Это связано с тем, что на период на период влиет лишь расстояние между ближайшими максимума, которые формируют эту картину, а при закрытии одного любого из них, расстояние между ближайшими не меняется.

\pagebreak

\section*{Заключение}
Были проведены измерения скорости ультразвука двумя разными способами:
\begin{enumerate}
    \item По дифракционной картине. Получены значения $v = 1525$ м/с, $v = 1489$ м/с, $v = 1550$ м/с, $v = 1506$ м/с и среднее $v = 1518$ м/с.
    \item Измерения методом темного поля. Получено значение $v = (179 \pm 11) \cdot 10 $ м/с.
\end{enumerate}
Видим, что значение, полученное методом темного поля, не сходится в пределах погрешности со средним значением, как и не сходится с некоторыми отдельными значениями. Это может объясняться тем, что при измерении по дифракционной картине относительно велика погрешность измерений, так как эти измерения основываются на измерениях расстояния между дифракционными максимумами, расстояние между которыми может быть неверно измерено из-за мелкости измеряемой картины. Тем не менее, полученное значение можно считать приемлимым.

\end{document}


\newcommand{\specialcell}[2][c]{%
	\begin{tabular}[#1]{@{}c@{}}#2\end{tabular}}

\newcommand{\mA}{\; мА}
\newcommand{\uA}{\; мкА}
\newcommand{\uV}{\; мкВ}
\newcommand{\V}{\; В}
\newcommand{\kV}{\; кВ}
\newcommand{\m}{\; м}
\newcommand{\del}{\; дел}
\newcommand{\mm}{\; мм}
\newcommand{\um}{\; мкм}
\newcommand{\nm}{\; нм}
\newcommand{\cm}{\; см}
\newcommand{\dptr}{\; дптр}
	
\begin{document}
\section*{Работа 4.7.2}	
\section*{Эффект Поккельса}
\subsection*{Киркича Андрей, Б01-202, МФТИ}		
\section*{Аннотация}
В работе мы, измерив радиусы интерференционных колец, определили разность показателей преломления $n_o - n_e$; подав на кристалл постоянное напряжение, получили свет, поляризованный по кругу; определили полуволновое напряжение по фигурам Лиссажу на экране осциллографа.
\section*{Экспериментальная установка}

Схема экспериментальной установки приведена на рисунке:

\begin{figure}[H]
	\centering
	\includegraphics[width=0.8\textwidth]{Изображения/ust.png}
	\caption{Схема эксперимента}
\end{figure}

\begin{figure}[H]
	\centering
	\includegraphics[width=0.8\textwidth]{Изображения/facility.png}
	\caption{Схема экспериментальной установки}
\end{figure}
\n
Параметры установки:
\begin{itemize}
\item Длина волны $\lambda = 630$ нм
\item Показатель преломления для обыкновенной волны $n_o = 2.29$
\item Длина кристалла $l = 25$ мм
\item Поперечный размер кристалла $d = 3$ мм
\item Цена деления шкалы прибора $1$ дел $= 15$ В.
\end{itemize}
\n
Луч света от лазера со встроенным вертикальным поляризатором попадает на кювету с кристаллом необата лития. Перед кюветой можно разместить матовую рассеивающую пластинку. Главная оптическая ось кристалла ориентирована вдоль направления распространения луча. После кюветы расположен поляроид. Результат интерференции наблюдается на экране. На кристалл можно подавать высоковольтное постоянное и переменное напряжение с помощью источника напряжения. Если подавать на кристалл переменное напряжение, то для исследования результата интерференции используется фотодиод, выход которого подключается к одному каналу осциллографа. Ко второму входу осциллографа подключается сигнал с источника напряжение.
	
\section*{Результаты измерений}

Сначала мы установили кристалл на расстоянии $L = 80.0 \pm 0.5$ см до экрана, осветили кристалл лазером через матовую пластину и получили на экране интерференционную картину. Лазерный свет поляризован в вертикальной плоскости. Радиусы тёмных колец $r_m$ представлены в таблице ниже. 

\begin{table}[H]
	\centering
\begin{tabular}{|r|r|r|r|r|r|r|}
	\hline
	$m$ & 1 & 2 & 3 & 4 & 5 & 6 \\
	\hline
	$r_m$, мм & 27.5 & 39.5 & 49.5 & 56.5 & 63.5 & 70.5 \\
	\hline
\end{tabular}
\end{table}
\n
По этим данным мы построили график зависимости квадрата радиуса тёмного кольца от его порядкового номера $r^2(m)$.

\begin{figure}[H]
	\centering
	\includegraphics[width=0.5\textwidth]{Графики/r2(m).png}
	\caption{График зависимости $r^2(m)$}
\end{figure}
\n
Данные хорошо сходятся с формулой:
\[r_m = \frac{\lambda}{l} \frac{(n_o L)^2}{n_o - n_e} m\]
\n
Зависимость можно аппроксимировать прямой $y = a x + b$, по углу наклона определить двулучепреломление $n_o - n_e$: \\
\[a = 835 \pm 12 \text{ мм}^2\]
\[b = -95 \pm 47 \text{ мм}^2\]

$$n_o - n_e = \frac{\lambda}{l}\frac{(n_o L)^2}{a} = 0.097 \pm 0.002$$

\subsection*{Определение полуволнового напряжения}
Убедившись, что направление лазерного луча совпадает с направлением на центр интерфереционной картины, мы убрали матовую пластинку. Подключили разъём блока питания на постоянно напряжение, установили регулятор напряжения на минимум и включили блок питания в сеть.
\n\n
Сначала мы определили интересующие нас напряжения без осциллографа. Для этого убрали матовую пластинку. При нулевом напряжении наблюдается минимум интенсивности излучения на экране. Постепенно увеличивая его, мы получали напряжения, соответствующие максимумам и минимумам интенсивности.
\n\n
При перпендикулярных поляризациях лазера и анализатора:
\[U_{\lambda/2} = (270 \pm 15) \text{ В}, \qquad U_\lambda = (630 \pm 15) \text{ В}, \qquad U_{3\lambda/2} = (1020 \pm 15) \text{ В}\]
При паралелльных поляризациях:
\[U_{\lambda/2} = (240 \pm 15) \text{ В}, \qquad U_\lambda = (630 \pm 15) \text{ В}, \qquad U_{3\lambda/2} = (1110 \pm 15) \text{ В}\]
\n
Подав на кристалл напряжение $U_{\lambda/4} = \frac{1}{2}U_{\lambda/2}$ и вращая анализатор, мы убедились, что поляризация круговая.
\n\n
Дальнейшие измерения проводились при помощи осциллографа. Было определено полуволновое напряжение по разности напряжений при максимуме и минимуме у фигуры Лиссажу: $U_{\lambda/2} = \Delta U = 390 \pm 15 \text{ В}$. 
	
\section*{Обсуждение результатов и выводы}
В работе мы наблюдали интерференционную картину обыкновенной и необыкновенной волн, образовавшихся после прохождения кристалла ниобата лития монохроматического поляризованного лазерного излучения.
\n\n
Было измерено двулучепреломление кристалла в отсутствии внешнего электрического поля: 
\[n_o - n_e = 0.097 \pm 0.002.\]
\n
Согласно справочнику для длины волны $\lambda = 632.8$ нм кристалл ниобата лития имеет показатели преломления $n_o = 2.286$, $n_e = 2.203$. Тогда имеем:
\[(n_o - n_e)^{табл} = 0.083.\]
\n
Результаты не сходятся в пределах погрешности, но довольно близки. Расхождение можно объяснить несовершенством лазера - мощность была слабой и из-за этого оценка напряженией на максимумах и минимумах могла быть неточной.
\n\n
В работе было определено полуволновое напряжение образца кристалла ниобата лития $U_{\frac{\lambda}{2}} = (270 \pm 15)$ В методом наблюдения периодических изменений интенсивности света на экране при постоянном внешнем электрическом поле внутри пластинки. 
\n\n
Методом наблюдения фигур Лиссажу при переменном электрическом поле внутри пластинки мы получили значение $U_{\frac{\lambda}{2}} = (390 \pm 15)$ В.
\n\n
Результаты, полученные двумя разными методами, не совпадают. Это можно объяснить большой чуствительностью осциллографа к внешним воздействиям.
\end{document}